\documentclass[11pt]{article}
\usepackage{configuration/style}

\usepackage{graphicx}
\usepackage{subcaption}
\usepackage{float}
\usepackage{enumitem}

\title{Alma Mater Studiorum - Università di Bologna \\ LM Informatica \\ \textbf{Plain X - Project Management Report}}
\author{\textbf{X-Spark}\newline(Marco Ferrati, Michele Perlino, Tommaso Azzalin)}

\date{A.A. 2020/2021}
\begin{document}

\maketitle
\clearpage

\begin{abstract}
    Il documento corrente descrive la riprogettazione della dashboard \href{https://opendatadpc.maps.arcgis.com/apps/opsdashboard/index.html#/b0c68bce2cce478eaac82fe38d4138b1}{"COVID-19 Situazione Italia" del Dipartimento della Protezione Civile} 
    realizzata nell'ambito del progetto didattico "Plain X" dell'insegnamento di Usability \& User Experience Design LM Informatica presso l'Università di Bologna: 
    il sistema è stato ridisegnato in modo che sia specificatamente indirizzato ai giornalisti, al fine di semplificare la loro comprensione e 
    analisi dei dati relativi all'andamento della pandemia Covid-19.
    In particolare, verranno illustrati i singoli passaggi compiuti per ottenere la nuova interfaccia ed esperienza utente, per soddisfare i bisogni 
    del target di utenza di riferimento. 
\end{abstract}

\tableofcontents
\clearpage

%% Sezione 1: Ricerca etnografica
\subfile{chapters/ricerca-etnografica/ricerca-etnografica}
\clearpage

%% Sezione 2: Verifica delle risorse esistenti
\subfile{chapters/verifica-risorse-esistenti/verifica-risorse-esistenti}
\clearpage

%% Sezione 3: Studio di fattibilità
\subfile{chapters/studio-fattibilita/studio-fattibilita}
\clearpage

% Sezione 4: Proposta di design
\subfile{chapters/proposta-design/proposta-design}
\clearpage

% Sezione 5: Valutazione del design
\subfile{chapters/valutazione-design/valutazione-design}
\clearpage

% Sezione 6: Raccomandazioni finali
\subfile{chapters/raccomandazioni-finali/raccomandazioni-finali}
\clearpage


\end{document}
