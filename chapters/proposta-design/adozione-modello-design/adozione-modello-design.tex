\subsection{Adozione di un modello di design}

Abbiamo adottato il modello CAO=S perché riteniamo essere il più aderente alle nostre esigenze: essendo un gruppo costituito da tre profili informatici non avremmo soddisfatto la multidisciplinarietà indispensabile alla corretta adozione del modello ``Goal Oriented" di J. J. Garrett. Abbiamo scartato anche il modello ``ISO 9241-210", per il suo focus fondamentalmente incentrato sull'usabilità. L'aspetto primario su cui abbiamo veicolato il nostro impegno è il miglioramento dell'esperienza utente: il nostro fine infatti è quello di fidelizzare i giornalisti verso la nostra dashboard cosicché diventi la loro fonte di riferimento; sin dalle prime ore di lavoro, abbiamo ambito ad un'applicazione web la cui fruizione fosse soddisfacente e gratificante per i giornalisti e la cui consultazione permettesse loro di comprendere in maniera agevole e approfondita l'andamento dell'epidemia Covid-19.\\
Infine, il modello CAO=S richiede intrinsecamente un budget ridotto, per cui è il più adeguato per un progetto didattico. 