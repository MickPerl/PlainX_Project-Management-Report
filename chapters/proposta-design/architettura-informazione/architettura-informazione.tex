\section{Architettura dell'informazione}

In questa sezione, esplichiamo le scelte compiute circa l'organizzazione delle informazioni presentate dalla dashboard.\\
Seguendo i principi dell'\textit{Information Architecture}, abbiamo articolato le informazioni con l'obiettivo di permettere ai giornalisti di comprenderle in maniera efficace ed efficiente: in particolare, il formato con cui i dati sono presentati è affine alle loro necessità informative e alle attività che realizzano, con la conseguenza di una significativa riduzione dei tempi necessari alla loro analisi e alla scrittura degli articoli.\\
\noindent
Abbiamo riflettuto anche secondo l'\textit{Information Design}, nella consapevolezza che la maniera in cui i dati sono comunicati influenza la percezione degli utenti sul fenomeno in considerazione.

\subsubsection{Approccio}
Abbiamo deciso di adottare un approccio di tipo top-down per governare la complessa organizzazione di una dashboard che deve comunicare informazioni eterogenee e correlate fra loro.
Procedendo in questa maniera, possiamo, sin da subito, avere una visione d'insieme delle funzionalità che vanno fornite all'utente e le relazioni fra esse, evitando le sovrapposizioni che potrebbero presentarsi seguendo l'approccio bottom-up.
Inoltre, in un design necessariamente iterativo quale quello in considerazione, il top-down, partendo da una definizione provvisoria degli artefatti, ne supporta e favorisce la modifica e il raffinamento continuo.

\subsubsection{Organizzazione delle funzionalità e del contenuto}
Le caratteristiche dell'\textit{Information Architecture} da noi adottate sono le seguenti:
\begin{itemize}
	\item \textbf{Strutturazione}: abbiamo previsto elementi grafici interattivi che permettono all'utente di specificare il livello di granularità desiderato;
	\begin{itemize}
		\item mappa geografica: province $>$ regioni;
		\item evoluzioni temporali: giorno $>$ settimana $>$ 15 giorni $>$ mese;
	\end{itemize}
	\item \textbf{Classificazione}: dati aggregati, assoluti, relativi, serie temporali;
	\item \textbf{Organizzazione}: abbiamo previsto che ogni metrica venga visualizzata in componenti grafiche, organizzate in più schermate;
	\item \textbf{Ricercabilità}: sono assenti funzionalità di ricerca, essendo l'intero contenuto informativo già mostrato, mentre sono presenti invece numerose funzionalità di filtro;
	\item \textbf{Gestione}: abbiamo previsto la possibilità di configurare il layout delle varie componenti, al fine di elaborare visioni personalizzate.
\end{itemize}

\subsubsection{Struttura}
Abbiamo scelto di adottare una struttura di tipo \textit{Co-hierchical organization}, in quanto nella dashboard da noi progettata abbiamo deciso di avere più schermate allo stesso livello in cui una di queste è anche quella principale.