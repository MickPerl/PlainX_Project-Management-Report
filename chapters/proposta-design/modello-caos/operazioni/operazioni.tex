\subsubsection{Operazioni}
Di seguito vengono elencati le operazioni che vengono eseguite sui concetti del sistema.

\textit{Nota}: La operazione \textit{Remove} in tutti i concetti, escluso il concetto ``Personalizzazione dell'interfaccia", non è reso possibile perché la dashboard mostra serie temporali che hanno senso solo se tutti i dati facente parte di queste serie vengono conservati per intero.

\paragraph{Resoconto del giorno a livello nazionale}\mbox{}\\
\begin{itemize}
    \item \textit{Create}: la creazione avviene manualmente, e i valori di default sono 0 per tutte le proprietà, ad eccezione di data di riferimento che è di default impostata a quella corrente.
    La creazione deve avvenire in maniera singola poiché l'istanza è univoca e persistente nel tempo.
    Non viene fornita alcuna memoria per l'inserimento perché potrebbe portare ad errori di inserimento dell'istanza; i valori di default sono sufficienti.
    Dovendo essere l'istanza univoca, viene verificato che non vi siano già presenti istanze con la data di riferimento indicata e nel caso vi siano, viene notificato l'attore dell'operazione;
    \item \textit{View}: modalità di base \textit{Individual Reduced View} perché vengono mostrate immediatamente una serie di proprietà (data di riferimento, nuovi positivi, \dots) che sono ritenute dagli attori fondamentali.
    Altre ritenute meno importanti sono disponibili su richiesta, mostrando il concetto in modalità \textit{Full Individual View};
    \item \textit{Update}: è permessa solamente la modalità \textit{Specific Update} (la data di riferimento, una volta inserita, non può essere modificata);
    \item \textit{Remove}: non possibile.
\end{itemize}

\paragraph{Resoconto del giorno a livello regionale}\mbox{}\\
Vale quanto indicato per \textbf{Resoconto del giorno a livello nazionale} ma, nell'operazione \textit{Create}, il nome della regione è un dato necessario da inserire e non è possibile fornire un valore di default.
Nel caso in cui il nome della regione non fosse esistente l'attore verrebbe notificato.

\paragraph{Resoconto del giorno a livello provinciale}\mbox{}\\
\begin{itemize}
    \item \textit{Create}: come per \textit{Create} nei concetti \textbf{Resoconto del giorno a livello nazionale} e \textbf{Resoconto del giorno a livello regionale} ma, anche in questo caso, il nome della provincia è un dato necessario da inserire e di cui non si può fornire un valore di default.
    Nel caso in cui il nome della provincia non fosse esistente o non appartenesse alla regione indicata l'attore verrebbe notificato.;
    \item \textit{View}: date le poche proprietà, viene sfruttata la visualizzazione \textit{Full Individual View};
    \item \textit{Update}: come per \textit{Update} nei concetti \textbf{Resoconto del giorno a livello nazionale} e \textbf{Resoconto del giorno a livello regionale};
    \item \textit{Remove}: non possibile.
\end{itemize}

\paragraph{Situazione a livello nazionale}\mbox{}\\
Vale quanto indicato per \textbf{Resoconto del giorno a livello nazionale}.

\paragraph{Situazione a livello regionale}\mbox{}\\
Vale quanto indicato per \textbf{Resoconto del giorno a livello regionale}.

\paragraph{Situazione a livello provinciale}\mbox{}\\
Vale quanto indicato per \textbf{Resoconto del giorno a livello provinciale}.

\paragraph{Provvedimento}\mbox{}\\
\begin{itemize}
    \item \textit{Create}: la creazione di un'istanza di questo concetto avviene manualmente ogni qual volta un ente istituzionale pubblica un provvedimento per il contrasto del Covid-19.
    Non ci sono valori di default perché tutti i provvedimenti sono unici e non ha senso fornire valori iniziali.
    La creazione può avvenire in contemporanea per più istanze di più provvedimenti.
    Una volta creata l'istanza essa resta memorizzata.
    Non vi è alcun suggerimento di valori precedentemente inseriti per il carattere eterogeneo delle istanze.
    In caso sia già stato inserito un provvedimento con lo stesso nome o manchino alcune proprietà non è possibile la creazione dell'istanza.
    \item \textit{View}: viene utilizzata la visualizzazione \textit{Full Individual View}, in quanto le proprietà sono poche ed essenziali, quindi vanno mostrate per intero.
    \item \textit{Update}: è permessa la modalità \textit{Global Update} poiché ogni proprietà è modificabile;
    \item \textit{Remove}: non possibile.
\end{itemize}

\paragraph{Personalizzazione dell'interfaccia}\mbox{}\\
\begin{itemize}
    \item \textit{Create}: la creazione di una personalizzazione avviene implicitamente, apportando delle modifiche alle schermate della dashboard (ad esempio tramite drag \& drop delle componenti visive oppure la scelta di visualizzarne o nasconderne altre).
    È fornita una personalizzazione di default che è quella presentata al primo caricamento della dashboard ed è quella che viene presentata in Sezione 4.5.
    Ogni attore può elaborare una personalizzazione della dashboard alla volta.
    Il concetto è persistente.
    \item \textit{View}: la proprietà ``posizione delle componenti dell'interfaccia" è implicitamente visibile e la presenza/assenza è agevolmente mostrabile, di conseguenza viene sfruttata una visualizzazione ``Full Individual View";
    \item \textit{Update}: è permessa la modalità \textit{Global Update} poiché ogni proprietà è modificabile;
    \item \textit{Remove}: il concetto è eliminabile (non archiviabile).
\end{itemize}