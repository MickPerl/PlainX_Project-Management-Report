\subsection{Operazioni}
\label{ss:operazioni}
Di seguito sono elencate le operazioni eseguite sui concetti del sistema.\\
\noindent
\textit{Nota}: L'operazione \textit{Remove} in tutti i concetti, ad eccezione di ``Personalizzazione dell'interfaccia", non è permessa perché la dashboard mostra serie temporali che, per loro natura, hanno bisogno che tutti i valori siano conservati.

\subsubsection{Resoconto del giorno a livello nazionale}
\label{sss:operazioni-resoconto-del-giorno-livello-nazionale}
\begin{itemize}
    \item \textit{Create}: la creazione avviene manualmente e i valori di default sono 0 per tutte le proprietà, ad eccezione della data di riferimento che è impostata a quella corrente.
    La creazione deve avvenire in maniera singola poiché l'istanza è univoca e persistente nel tempo.
    Non viene fornita alcuna memoria per l'inserimento perché potrebbe portare ad errori: i valori di default sono sufficienti.
    Dovendo essere l'istanza univoca, viene verificata la presenza di istanze con la stessa data e, in caso affermativo, viene restituita una notifica di errore all'attore che ha svolto quell'operazione;
    \item \textit{View}: modalità di base \textit{Individual Reduced View} perché vengono mostrate immediatamente una serie di proprietà (data di riferimento, nuovi positivi, \dots) che sono ritenute fondamentali dagli attori.
    Altre considerate meno importanti sono disponibili su richiesta e mostrano il concetto in modalità \textit{Full Individual View};
    \item \textit{Update}: è permessa solamente la modalità \textit{Specific Update} (la data di riferimento, una volta inserita, non può essere modificata);
    \item \textit{Remove}: non possibile.
\end{itemize}

\subsubsection{Resoconto del giorno a livello regionale}
\label{sss:operazioni-resoconto-del-giorno-livello-regionale}
Vale quanto indicato in \hyperref[sss:operazioni-resoconto-del-giorno-livello-nazionale]{Resoconto del giorno a livello nazionale} ma, nell'operazione \textit{Create}, il nome della regione è un dato necessario da inserire e non è possibile fornire un valore di default.
Nel caso in cui il nome della regione non fosse esistente, viene restituita una notifica di errore all'attore.

\subsubsection{Resoconto del giorno a livello provinciale}
\label{sss:operazioni-resoconto-del-giorno-livello-provinciale}
\begin{itemize}
    \item \textit{Create}: come per \textit{Create} in \hyperref[sss:operazioni-resoconto-del-giorno-livello-nazionale]{Resoconto del giorno a livello nazionale} e \hyperref[sss:operazioni-resoconto-del-giorno-livello-regionale]{Resoconto del giorno a livello regionale}, eccetto che per il nome della provincia che è un dato necessario da inserire e di cui non si può fornire un valore di default.
    Nel caso in cui il nome della provincia non esistesse o non appartenesse alla regione indicata, viene restituita una notifica di errore all'attore;
    \item \textit{View}: date le poche proprietà, viene sfruttata la visualizzazione \textit{Full Individual View};
    \item \textit{Update}: come per \textit{Update} in \hyperref[sss:operazioni-resoconto-del-giorno-livello-nazionale]{Resoconto del giorno a livello nazionale} e \hyperref[sss:operazioni-resoconto-del-giorno-livello-regionale]{Resoconto del giorno a livello regionale};
    \item \textit{Remove}: non possibile.
\end{itemize}

\subsubsection{Situazione a livello nazionale}
\label{sss:operazioni-situazione-livello-nazionale}
Vale quanto indicato in \hyperref[sss:operazioni-resoconto-del-giorno-livello-nazionale]{Resoconto del giorno a livello nazionale}.

\subsubsection{Situazione a livello regionale}
\label{sss:operazioni-situazione-livello-regionale}
Vale quanto indicato in \hyperref[sss:operazioni-resoconto-del-giorno-livello-regionale]{Resoconto del giorno a livello regionale}.

\subsubsection{Situazione a livello provinciale}
\label{sss:operazioni-situazione-livello-provinciale}
Vale quanto indicato in \hyperref[sss:operazioni-resoconto-del-giorno-livello-provinciale]{Resoconto del giorno a livello provinciale}.

\subsubsection{Provvedimento}
\label{sss:operazioni-provvedimento}
\begin{itemize}
    \item \textit{Create}: la creazione di un'istanza avviene manualmente ogni qual volta un ente istituzionale emana un provvedimento in materia di contrasto della pandemia.
    Non ci sono valori di default perché tutti i provvedimenti sono unici e non ha senso fornire valori iniziali.
    La creazione può avvenire in contemporanea per più istanze di più provvedimenti.
    Una volta creata l'istanza, resta memorizzata.
    Non vi è alcun suggerimento di valori precedentemente inseriti per il carattere eterogeneo delle istanze.
    In caso sia già stato inserito un provvedimento con lo stesso nome o manchino alcune proprietà, la creazione viene impedita.
    \item \textit{View}: viene utilizzata la visualizzazione \textit{Full Individual View}, in quanto le proprietà sono poche ed essenziali, quindi vanno mostrate per intero;
    \item \textit{Update}: è permessa la modalità \textit{Global Update} poiché ogni proprietà è modificabile;
    \item \textit{Remove}: non possibile.
\end{itemize}

\subsubsection{Personalizzazione dell'interfaccia}
\label{sss:operazioni-personalizzazione-interfaccia}
\begin{itemize}
    \item \textit{Create}: la creazione di una personalizzazione avviene implicitamente, apportando delle modifiche alle schermate della dashboard (ad esempio tramite drag \& drop delle componenti visive oppure la scelta di visualizzarne o nasconderne altre).
    È fornita una personalizzazione di default che è quella presentata al primo caricamento della dashboard ed è quella che viene presentata nella \hyperref[s:struttura-blueprint]{Sezione 4.5 Blueprint}.
    Ogni attore può elaborare una personalizzazione della dashboard alla volta.
    Il concetto è persistente.
    \item \textit{View}: la proprietà ``posizione delle componenti dell'interfaccia" è implicitamente visibile e la presenza/assenza è visualizzabile agevolmente, per cui viene sfruttata una visualizzazione ``Full Individual View";
    \item \textit{Update}: è permessa la modalità \textit{Global Update} poiché ogni proprietà è modificabile;
    \item \textit{Remove}: il concetto è eliminabile (non archiviabile).
\end{itemize}