\subsubsection{Concetti}
I concetti sono stati più volte rielaborati per raggiungere la loro forma finale.
In particolare, la prima loro definizione conteneva solo proprietà pure, non inclusive di dati aggregati. Successivamente sono stati aggiunti dati aggregati come i tassi, le percentuali, i saldi.
Ancora, sono stati aggiunte proprietà quali la data di riferimento e i nomi delle regioni/provincie.
In seguito, è stato aggiunto il concetto ``Provvedimento" che rappresenta un provvedimento del Presidente del Consiglio, del Governo e tutti gli altri enti aventi potere di legiferazione o emanazione di atti amministrativi.
In battuta finale, abbiamo aggiunto anche i dati anagrafici, quali l'età e il genere, nei concetti Situazione a livello nazionale/regionale/provinciale.

Di seguito vengono elencati i concetti che vengono manipolati dagli attori che utilizzano il sistema.
\paragraph{Resoconto del giorno a livello nazionale}\mbox{}\\
Rappresenta le informazioni sull'evoluzione della pandemia a livello nazionale relative alla data di interesse.\\
È caratterizzato dalle seguenti proprietà:
\begin{itemize}
    \item data di riferimento;
    \item nuovi positivi;
    \item nuovi decessi;
    \item nuovi ingressi in terapia intensiva;
    \item nuovi tamponi;
    \item nuovi dimessi/guariti (una persona dimessa dall'ospedale ma non guarita viene inserita in questo conteggio, con l'ipotesi che prima o poi avverrà la guarigione);
    \item tasso di positività = $\frac{nuovi \; positivi}{nuovi \; tamponi}$;
    \item saldo terapia intensiva = totale terapie intensive alla data di riferimento - totale terapie intensive al giorno precedente a quello di riferimento;
    \item note sui dati pubblicati;
\end{itemize}
In una precedente iterazione questo concetto era stato denominato ``Andamento giornaliero nazionale" ma, potendo confondersi con alcune parti dell'interfaccia presentate nella Sezione 4.5, abbiamo ritenuto necessario di rinominarlo.

\paragraph{Resoconto del giorno a livello regionale}\mbox{}\\
Rappresenta le informazioni sull'evoluzione della pandemia a livello regionale (ovvero di una singola regione) relative alla data di interesse.\\
È caratterizzato dalle seguenti proprietà:
\begin{itemize}
    \item (tutte le proprietà definite per \textbf{Resoconto del giorno a livello nazionale} ma riferiti alla singola regione);
    \item nome della regione;
    \item colore della regione secondo i DPCM 03/11/2020 e 16/01/2021.
\end{itemize}
In una precedente iterazione questo concetto era stato denominato ``Andamento giornaliero regionale" ma, per le stesse ragioni per cui è stato rinominato il concetto ``Resoconto del giorno a livello nazionale", è stato modificato anche questo.

\paragraph{Resoconto del giorno a livello provinciale}\mbox{}\\
Rappresenta le informazioni sull'evoluzione della pandemia a livello provinciale (ovvero di una singola provincia) relative alla data di interesse.\\
È caratterizzato dalle seguenti proprietà:
\begin{itemize}
    \item data di riferimento;
    \item nome della regione;
    \item nome della provincia;
    \item nuovi positivi.
\end{itemize}
In una precedente iterazione questo concetto era stato denominato ``Andamento giornaliero provinciale" ma, per le stesse ragioni per cui sono stato rinominati i concetti ``Resoconto del giorno a livello nazionale" e ``Resoconto del giorno a livello regionale", è stato modificato anche questo.

\paragraph{Situazione a livello nazionale}\mbox{}\\
Rappresenta la situazione epidemiologica in Italia ad oggi, ossia il cumulato dei dati a partire dall'inizio della pandemia.\\
È caratterizzato dalle seguenti proprietà:
\begin{itemize}
    \item data di riferimento;
    \item totale casi;
    \item totale casi ogni 100K abitanti = $\frac{totale \; casi \; \cdot \; 100.000}{totale \; abitanti}$;
    \item totale decessi;
    \item totale decessi ogni 100K abitanti = $\frac{totale \; decessi \; \cdot \; 100.000}{totale \; abitanti}$;
    \item totale terapie intensive (ovvero il numero di pazienti attualmente in terapia intensive);
    \item totale ospedalizzati  (ovvero il numero di pazienti attualmente in ospedale per ottenere cure legate al virus);
    \item totale ospedalizzati ogni 100K abitanti = $\frac{totale \; ospedalizzati \; \cdot \; 100.000}{totale \; abitanti}$;
    \item tamponi totali;
    \item totale dimessi/guariti;
    \item totale in isolamento domiciliare (ovvero il numero di persone attualmente positive e non in ospedale, di cui vengono conteggiate solamente coloro che non hanno avuto cure da parte degli ospedali, i dimessi ma non ancora guariti non sono inseriti in questo conteggio);
    \item totale in isolamento domiciliare ogni 100K abitanti = $\frac{totale \; isolati \; nel \; proprio \; domicilio \; \cdot \; 100.000}{totale \; abitanti}$;
    \item attuali positivi (ossia quanti sono ancora positivi al tampone molecolare);
    \item \% attuali positivi sull'intera popolazione;
    \item attuali positivi ogni 100K abitanti = $\frac{nuovi \; positivi \; \cdot \; 100.000}{totale \; abitanti}$;
    \item totale positivi maschi;
    \item \% positivi maschi sul totale casi;
    \item totale positivi femmine;
    \item \% positivi femmine sul totale casi;
    \item totale casi per fasce d'età (0-18, 19-50, 51-70, $>$ 70);
    \item totale casi tra operatori sanitari;
    \item tasso di letalità = $\frac{totale \; decessi}{totale \; casi}$;
    \item RT (il dato indica la probabilità di trasmissione del virus che rapporta gli attuali sintomatici con i possibili nuovi contagi; se supera il valore 1 significa che il virus si sta diffondendo e i casi stanno aumentando)\footnote{Da \href{https://tg24.sky.it/salute-e-benessere/approfondimenti/covid-indice-rt}{https://tg24.sky.it/salute-e-benessere/approfondimenti/covid-indice-rt}};
    \item tasso occupazione terapie intensive = $\frac{totale \; terapie \; intensive}{posti \; totali \; disponibili \; in \; terapia \; intensiva}$.
    \item note sui dati pubblicati.
\end{itemize}
In corrispondenza della proprietà ``Attuali positivi" abbiamo riscontrato un problema di standardizzazione: gli attori diretti, quali i giornalisti, usano l'espressione ``Attuali positivi" mentre l'attore indiretto rivestito dall'Amministratore della base di dati è solito riferirsi alla medesima proprietà con l'espressione di ``totale\_casi". Abbiamo risolto questa discrepanza, adottando l'espressione utilizzata dall'attore diretto nella consapevolezza che, per definizione, è lui a consultare direttamente la dashboard.

\paragraph{Situazione a livello regionale}\mbox{}\\
Rappresenta la situazione epidemiologica in una regione ad oggi, ossia il cumulato dei dati a partire dall'inizio della pandemia.\\
È caratterizzato dalle seguenti proprietà:
\begin{itemize}
    \item (tutte le proprietà definite per \textbf{Situazione a livello nazionale} ma riferite alla singola regione);
    \item nome della regione.
\end{itemize}

\paragraph{Situazione a livello provinciale}\mbox{}\\
Rappresenta la situazione epidemiologica in una provincia ad oggi, ossia il cumulato dei dati a partire dall'inizio della pandemia.\\
È caratterizzato dalle seguenti proprietà:
\begin{itemize}
    \item data di riferimento;
    \item nome della regione;
    \item nome della provincia;
    \item totale casi registrati ad oggi.
\end{itemize}

\paragraph{Provvedimento}\mbox{}\\
Rappresenta un provvedimento emanato dal Presidente del Consiglio, dal Governo o dai Ministeri, dal Parlamento, dai Consigli Regionali o Comunali.\\
È caratterizzato dalle seguenti proprietà:
\begin{itemize}
    \item denominazione;
    \item ente di pubblicazione;
    \item territorio di riferimento;
    \item intervallo di validità (date di inizio e fine);
    \item riferimento esterno al documento.
\end{itemize}

\paragraph{Personalizzazione dell'interfaccia}\mbox{}\\
Rappresenta quali parti dell'interfaccia sono correntemente visualizzate all'utente.\\
È caratterizzato dalle seguenti proprietà:
\begin{itemize}
    \item posizione delle componenti nell'interfaccia;
    \item presenza o assenza delle componenti nell'interfaccia.
\end{itemize}