\section{Progettazione dell'interazione}
\label{s:approccio-interaction-design}

\subsection{Ergonomia}
\label{ss:ergonomia}

\subsubsection{Organizzazione dei controlli e del display}
\label{sss:organizzazione-controlli-display}
La nostra riprogettazione della dashboard del DPC vede un'organizzazione dei controlli e delle schermate secondo il raggruppamento per frequenza: abbiamo adottato questa scelta perché riteniamo sia idonea a velocizzare il lavoro dei giornalisti che vi operano, i quali potranno, con rapide occhiate, visionare e interagire con le componenti grafiche che utilizzano più di frequente.

\subsubsection{Condizioni ambiente fisico}
\label{sss:condizioni-ambiente-fisico}
La nostra riprogettazione avrà come output un'applicazione web-based, per cui non possiamo impattare sulle condizioni dell'ambiente fisico in cui gli utenti ne fruiscono.

\subsubsection{Uso dei colori}
\label{sss:uso-colori}
Nella riprogettazione compiuta, abbiamo prestato particolare attenzione all'utilizzo dei colori: crediamo sia un aspetto essenziale per i suoi riflessi sull'usabilità complessiva del sistema. In dettaglio, abbiamo cercato di evitare situazioni in cui i colori siano l'unico mezzo per distinguere certe componenti o informazioni, al fine di garantire una buona usabilità agli utenti con eventuali difficoltà nella distinzione delle tonalità (soggetti anziani, daltonici).\\
Per verificare di aver realizzato correttamente le nostre intenzioni, abbiamo previsto di testare la nostra riprogettazione con un soggetto daltonico (deuteranopia, tipo di daltonismo).  

\subsection{Progettazione della conversazione}
\label{ss:progettazione-conversazione}
Nella nostra riprogettazione, abbiamo adottato principalmente 2 stili di interazione: trattasi del ``Menu e Navigazione" e ``Manipolazione diretta", ritenendoli i più appropriati per l'interazione esperita con una dashboard e i più efficaci nel supportare i task dei giornalisti.

\subsubsection{Menu e Navigazione}
\label{sss:menu-navigazione}
I vantaggi apportati dallo stile ``Menu e Navigazione", quali la corta curva di apprendimento, il basso numero di tipologie di azioni, la strutturazione dei task dell'utente e la gestione semplice degli errori, riteniamo essere particolarmente significativi per i nostri fini.
--INTEGRARE DOPO PROGETTAZIONE--

\subsubsection{Manipolazione diretta}
\label{sss:manipolazione-diretta}
Lo stile della ``Manipolazione diretta" è spesso adottato per la semplicità di apprendimento e ricordo che rende possibile realizzare nell'interazione, oltre che per la presentazione visiva dei concetti che favorisce la comprensione dell'utente: il nostro redesign presenta numerosi grafici e componenti visive, per cui non può che trarre grandi benefici da questo stile di interazione. Inoltre, l'utente, grazie ad esso, è messo nelle condizioni di poter esplorare il contenuto informativo della dashboard e acquisire informazioni nuove, anche senza averle richieste espressamente (serendipità\footnote{Il termine serendipità indica l'occasione di fare felici scoperte per puro caso e, anche, il trovare una cosa non cercata e imprevista mentre se ne stava cercando un'altra.}).\\
Ancora, crediamo che la possibilità di interagire coi grafici offerta ai giornalisti giochi un ruolo fondamentale nel migliorarne l'esperienza di utilizzo, con benefici sulla loro personale soddisfazione complessiva.\\
D'altro canto, questo stile, tra gli svantaggi, richiede uno schermo grafico e un sistema di puntamento: il risultato della nostra riprogettazione è un'applicazione web-based, la quale, per definizione, viene fruita tramite un browser e quindi un computer, andando a soddisfare entrambi i requisiti di cui sopra.\\
Al fine di presentare all'utente un sistema familiare, abbiamo deciso di scegliere per il nostro sito il layout di una dashboard, avvicindosi esso a quello di un cruscotto di un'autovettura ovvero di una plancia di comando di una nave, entrambi oggetti del mondo reale.\\
Per quanto riguarda la direttezza (\textit{directness}) nella manipolazione abbiamo optato per la tipologia articolatoria, in quanto vi sono relazioni dirette tra le funzionalità del sistema e i comandi che le attivano: ad esempio, vi è la funzionalità di filtro dei dati presentati a livello di una o più specifiche regioni, la quale è attivata mediante l'interazione del giornalista con componenti grafiche che saranno definite compiutamente nel wireframe.

\subsection{Progettazione delle schermate}
\label{ss:progettazione-schermate}
--INTEGRARE DOPO--
Ridondanza box con metriche giornaliere e grafici temporali cumulati.

\subsection{Studio del contesto sociale e della organizzazione}
\label{ss:studio-contesto-sociale-organizzazione}
La nostra riprogettazione ha l'obiettivo di fornire al giornalista un sistema tramite cui possa soddisfare agevolmente i propri bisogni informativi, nonché presentargli una visione integrata delle fonti istituzionali dei dati epidemiologici. Crediamo che una dashboard usabile e dalla soddisfacente fruibilità favorisca la riduzione degli errori nella comprensione dei fenomeni e porti ad una informazione corretta e più completa: inoltre, la funzionalità di condivisione di layout personalizzati si concretizza in una collaborazione più efficace tra i giornalisti.
