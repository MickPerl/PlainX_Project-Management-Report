\documentclass[../../main]{subfiles}

\begin{document}

\chapter{Raccomandazioni finali}
\label{c:raccomandazioni-finali}

\let\oldquote\quote
\let\endoldquote\endquote
\renewenvironment{quote}[2][]
{\if\relax\detokenize{#1}\relax
        \def\quoteauthor{#2}%
    \else
        \def\quoteauthor{#2~---~#1}%
    \fi
    \oldquote}
{\par\nobreak\smallskip\hfill(\quoteauthor)%
    \endoldquote\addvspace{\bigskipamount}}

\begin{quote}{M.T. Island}
    In questi mesi ci siamo affidati ai numeri per avere risposte asettiche, prive di condizionamenti.
    Tuttavia anche i numeri possono mentire: se li costringiamo a farlo, per esempio, ma anche se li guardiamo inquadrandoli nel contesto sbagliato.
    Ed è quello che, in molti casi, sta accadendo con l'interpretazione della fase attuale dell'epidemia di Covid-19: non sempre i numeri sono raccolti ed esposti in modo corretto, e soprattutto non li analizziamo nel giusto contesto.
\end{quote}
\noindent
L'interfaccia frutto della nostra riprogettazione ha raggiunto l'obiettivo di fornire il giusto contesto alle analisi dei giornalisti, così da permetter loro di maturare comprensioni corrette e profonde della metriche epidemiologiche.\\
In dettaglio, siamo partiti conducendo una ricerca etnografica che ci ha permesso di raccogliere le esigenze e le testimonianze dei giornalisti, in qualità di utenti diretti dell'interfaccia.
In seguito, abbiamo individuato linee guida ritenute di valore nella valutazione di dashboard e le abbiamo analizzate sull'interfaccia originale, giungendo così alla raccolta di una serie di criticità.
Siamo passati, quindi, a studiare la fattibilità della nostra riprogettazione, definendo il contesto d'uso, i task e le persona da supportare, nonché gli scenari in cui credibilmente operano.
A questo punto, siamo entrati nel vivo della definizione della nostra riprogettazione, effettuando scelte relativamente all'architettura delle informazioni e dell'interazione da realizzare e seguendo il modello CAO=S: il frutto di questi studi sono stati i blueprint e i wireframe.
Infine, abbiamo condotto l'ispezione e il testing con degli utenti reali, raccogliendo insight circa le criticità risolte e gli aspetti ancora da raffinare.
Ogni fase ha visto susseguirsi diverse iterazioni, necessarie per giungere alla versione finale dell'interfaccia.\\
Sulla base di ciò, siamo fiduciosi che la nostra riprogettazione sia di qualità e possa apportare grandi benefici sulla comprensione della pandemia in atto da parte dei giornalisti o chiunque ne sia interessato: con riferimento all'interfaccia originaria della dashboard, quella da noi proposta offre ai suoi utenti facilità di utilizzo, flessibilità, un'esperienza d'uso proficua e soddisfacente.
\end{document}