\subsubsection{Giornalista testata nazionale}
\label{sss:giornalista-testata-nazionale}
\begin{wrapfigure}{r}{8cm}
    \includegraphics[scale=0.3]{studio-fattibilità/giovanni}
    \caption{\raggedright Foto fantasiosa della persona Giovanni}
\end{wrapfigure}
Giovanni è un uomo di 53 anni di Fiumicino. È felicemente sposato da 24 anni e ha due figlie di 19 e 22 anni.\\ 
Dopo aver completato gli studi presso l'Università La Sapienza di Roma e aver fatto della gavetta presso qualche giornale locale, lavora ormai da 19 anni presso il \textit{Corriere della Sera}, dove ricopre il ruolo di capo redattore. Nell'ultimo anno si è occupato di scrivere articoli per informare i lettori sull'andamento della pandemia dovuta al Covid-19. Ultimamente ha ripreso a lavorare in redazione assieme ad altre 10 persone, ognuno con la propria personale postazione correttamente distanziata dalle altre, in cui è presente un computer fisso con un monitor da 24".
Per scrivere i suoi articoli utilizza un software proprietario, sviluppato per l'azienda, di cui ormai conosce la maggior parte delle funzionalità. Nell'ultimo periodo ha imparato ad utilizzare la dashboard del DPC per raccogliere i dati che inserisce nei suoi articoli.
\begin{itemize}
	\item Attitudine:
	\begin{itemize}
        \item esperto con tecnologie informatiche per cui richiede a suo supporto sistemi configurabili secondo le sue esigenze;
        \item è pignolo e esige che gli giungano informazioni in maniera rigorosa e tempestiva;
        \item è oberato di impegni, per cui ha necessità di essere supportato nel suo lavoro da strumenti altamente produttivi e affidabili;
    \end{itemize}
    \item Comportamento: per comprendere i dati sul Covid-19, usa solo dashboard (no articoli, interviste, ecc.) che siano:
	\begin{itemize}
	    \item agevoli nella comprensione;
	    \item affidabili nel funzionamento;
    \end{itemize}
	\item Obiettivi (\textit{end goals}): scrittura rapida di articoli riguardanti l'andamento dell'epidemia, ricavando i dati dalla dashboard del DPC per fornire un'analisi significativa;
	\item Motivazione (\textit{life goals}): fornire informazioni rigorose ai suoi lettori, con particolar attenzione all'analisi dei dati per permettere loro di comprendere approfonditamente la situazione nel contesto, così da prevenire allarmismi o sottostime del fenomeno;
	\item Obiettivi del sistema:
	\begin{itemize}
	    \item sistema di notifiche in caso di aggiornamento dei dati;
	    \item configurabilità interfaccia;
	    \item tracciabilità dei dati (\textit{data lineage}\footnote{La tecnica del \textit{data lineage} consiste nell’identificazione e rappresentazione del ciclo di vita del dato.}).
    \end{itemize}
\end{itemize}