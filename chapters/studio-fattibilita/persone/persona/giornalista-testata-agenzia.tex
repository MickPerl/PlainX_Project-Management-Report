\subsubsection{Giornalista testata d'agenzia}
\begin{wrapfigure}{r}{7cm}
    \includegraphics{studio-fattibilità/giulia}
    \caption{Foto fantasiosa della persona Giulia}
\end{wrapfigure}
Giulia è una donna di 35 anni, laureata in Scienze Politiche all'Università di Milano.\\ 
Grande appassionata di tecnologia, si informa su come essa possa semplificarle la vita: infatti, a casa, ha un'assistente vocale e numerose luci smart, controllabili tramite il suo smartphone.
Convive da otto anni con il suo fidanzato in un piccolo appartamento nel centro di Milano, a 10 minuti a piedi dall'ufficio in cui lavora. Condivide l'ufficio assieme ad altre 4 persone, lavorando sul portatile datole dall'agenzia di stampa con il quale può lavorare anche da casa. L'agenzia richiede che il lavoro sia svolto principalmente in smart working finché la situazione epidemiologica non migliora: prevede una deroga per stretta necessità per cui ci si può recare in ufficio.
Il suo ruolo nell'agenzia è quello di ``fact-checking", ossia raccogliere i dati relativi ad un certo fenomeno e verificarne la veridicità, in modo che non vengano pubblicati dati errati o incompleti.\\ 
Nell'ultimo periodo i dati che maggiormente ha studiato sono quelli relativi all'epidemia Covid-19. La sua fonte principale per questo tema è la dashboard del DPC ma fa riferimento in maniera rigorosa e sistematica anche alle fonti riferite in essa. Per il suo lavoro si trova spesso a ricostruire il ciclo di vita di ogni dato così da verificarne la veridicità.\\ 
Per rimanere in comunicazione con i colleghi utilizza chat di gruppo, sistemi di collaborazione e file sharing, riuscendo a destreggiarsi con agevolezza tra le funzionalità che essi offrono. 
\begin{itemize}
	\item Attitudine:
    \begin{itemize}
        \item rigore nella verifica della veridicità delle fonti delle informazioni;
        \item predisposizione a lavorare in maniera rapida, altamente produttiva, disposta ad utilizzare strumenti tecnologici di automatizzazione;
    \end{itemize}
    \item Comportamento: 
    \begin{itemize}
        \item verifica sempre le fonti, ricostruendo il \textit{data lineage};
        \item aggiorna tempestivamente eventuali dati errati/incompleti;
        \item valuta immediatamente le segnalazioni in arrivo da altri presidi di informazione;
    \end{itemize}
    \item Obiettivi (\textit{end goals}): comunicare in maniera puntuale e tempestiva i dati visualizzati nella dashboard del DPC;
    \item Motivazione (\textit{life goals}): informare le altre testate giornalistiche circa i dati veritieri e verificati, così da costituire per loro un punto di riferimento credibile e autorevole;
    \item Obiettivi del sistema:
    \begin{itemize}
        \item notifiche;
        \item link alle fonti.
    \end{itemize}
\end{itemize}