\subsubsection{Giornalista testata locale}
\label{sss:giornalista-testata-locale}
\begin{wrapfigure}{l}{0.4\textwidth}
	\centering
	\vspace{-13pt}
    \includegraphics[width=0.4\textwidth]{studio-fattibilità/francesca}
    \caption{Foto fantasiosa della persona Francesca}
\end{wrapfigure}
Francesca è una donna di 36 anni di Mestre, laureatasi in Scienze della Comunicazione all'Università di Padova.\\ 
Lavora da sette anni presso la redazione del Gazzettino, in cui cerca di raccontare la propria regione e i suoi cittadini con passione, per fornire il proprio contributo diventato ancora più importante durante questa pandemia.\\ 
Essendo nella sua regione la situazione particolarmente critica, la direzione del giornale ha deciso di far lavorare da remoto e in presenza i suoi dipendenti a settimane alterne, mantenendo invariati i gruppi che lavorano in presenza presso gli uffici in una determinata settimana in modo da poter prevenire delle possibili diffusioni del contagio.\\Il contatto fisico è ridotto al minimo, tantoché tutte le comunicazioni avvengono rigorosamente tramite lo strumento di messaggistica \textit{Slack}.\\ 
Per via del suo percorso formativo e della passione per il suo territorio, ha a cuore il raccontare ai propri concittadini come i numeri che tutti i giorni sentono in televisione e leggono sui giornali influenzino le loro vite. Per questo motivo, oltre all'analisi dei dati provenienti dalla dashboard del DPC e dai bollettini regionali, intervista i sindaci delle città venete per comprendere le strategie intraprese al fine di fronteggiare l'emergenza e raccoglie le istanze dei piccoli imprenditori locali.
\begin{itemize}
    \item Attitudine:
    \begin{itemize}
        \item approccio non rigoroso ai dati, per cui si fida delle dichiarazioni dei media pubblici e privati;
        \item sguardo ristretto alla propria realtà cittadina/regionale, basso interesse per lo scenario nazionale;
    \end{itemize}
	\item Comportamento: 
	\begin{itemize}
	    \item consulta dashboard sul Web, legge articoli e dichiarazioni dei sindaci e del Presidente di Regione;
	    \item intervista sindaci e imprenditori locali per ascoltare le loro opinioni o prospettive future;
    \end{itemize}
	\item Obiettivi (\textit{end goals}): raccontare come il Covid-19 abbia influenzato e continui a influenzare la realtà in cui vive e come il territorio reagisce, in termini di:
	\begin{itemize}
        \item nuove sfide per la società (economiche, sociali, ecc.),
        \item criticità che possano coinvolgere i suoi cittadini (chiusura piazze, corsi, centro commerciali, ecc.),
        \item contatti dei presidi sanitari/assistenziali locali e le buone pratiche da adottare;
    \end{itemize}
	\item Motivazione (\textit{life goals}): rendere i suoi concittadini consapevoli dell'evoluzione della pandemia nel loro territorio;
	\item Obiettivi del sistema:
    \begin{itemize}
        \item Possibilità di ricontestualizzare tutti gli elementi comunicativi dal livello nazionale a quello regionale (filtro per regione);
        \item Presentazione delle ordinanze regionali/provinciali/comunali, con focus su quelle della regione di interesse;
        \item Presentazione dei provvedimenti nazionali che hanno impatto sulla regione di interesse.
    \end{itemize}
\end{itemize}