\subsection{Persona}
Abbiamo individuato e definito dettagliatamente sei \textit{persona}\footnote{Con il termine \textit{persona} si intende un utente sintetico ideato in modo che incarni tutte gli aspetti (intenzioni, obiettivi, attitudini, comportamenti, abitudini...) differenti che si vogliono supportare nell'ambito del design di un prodotto.}, ciascuna raffigurante un profilo interessante e peculiare. 
\noindent
In una prima iterazione, nella selezione delle \textit{persona} avevamo incluso anche i giornalisti di testate esclusivamente online, per poi accorgerci che presentavano caratteristiche del tutto assimilabili alla \textit{persona} del blogger e del giornalista di testate nazionali o locali. \\ 
Inoltre, in una prima versione, avevamo selezionato la \textit{persona} ``cittadino", ma, nel definirla, ci siamo accorti della possibilità di articolarla in due \textit{persona} distinte, quella del ``cittadino esperto" e quella dell'``utente che utilizza smartphone".\\
Dall'analisi più approfondita delle dashboard preesistenti, ci siamo resi conto della presenza importante di colori come strumento per distinguere gli elementi grafici: essendo per tali applicazioni web di assoluta rilevanza il ``colpo d'occhio", crediamo che affidarsi solamente alla capacità distintiva dei colori sia insufficiente, considerando le persone affette da daltonismo. Pertanto, un'ulteriore iterazione ci ha portato a caratterizzare maggiormente la \textit{persona} Roberto, esplicitando la sua difficoltà nel riconoscere determinate tonalità.
\noindent
Di seguito riportiamo le \textit{persona} cui siamo giunti, articolandole per tipologia e presentando per ognuna una foto, un elenco puntato delle informazioni salienti richieste e una descrizione discorsiva.
\newpage
\subsubsection{Protagonisti}

\subfile{persona/giornalista-testata-nazionale}

\subfile{persona/giornalista-testata-locale}

\subfile{persona/giornalista-testata-agenzia}

\subsubsection{Persona secondarie}

\subfile{persona/blogger}

\subsubsection{Persona addizionali}

\subfile{persona/cittadino-esperto}

\subsubsection{Persona non-utenti, negative}

\subfile{persona/utente-dispositivo-mobile}