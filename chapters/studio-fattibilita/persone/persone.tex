\subsection{Persona}
Abbiamo individuato e definito dettagliatamente sei \textit{persona}\footnote{Con il termine \textit{persona} si intende un utente sintetico ideato in modo che incarni tutte gli aspetti (intenzioni, obiettivi, attitudini, comportamenti, abitudini...) differenti che si vogliono supportare nell'ambito del design di un prodotto.}, ciascuna raffigurante un profilo interessante e peculiare. 
\noindent
In una prima iterazione, nella selezione delle \textit{persona} avevamo incluso anche i giornalisti di testate esclusivamente online, per poi accorgerci che presentavano caratteristiche del tutto assimilabili alla \textit{persona} del blogger e del giornalista di testate nazionali o locali. \\ 
Inoltre, in una prima versione, avevamo selezionato la \textit{persona} ``cittadino", ma, nel definirla, ci siamo accorti della possibilità di articolarla in due \textit{persona} distinte, quella del ``cittadino esperto" e quella dell'``utente che utilizza smartphone".\\
Dall'analisi più approfondita delle dashboard preesistenti, ci siamo resi conto della presenza importante di colori come strumento per distinguere gli elementi grafici: essendo per tali applicazioni web di assoluta rilevanza il ``colpo d'occhio", crediamo che affidarsi solamente alla capacità distintiva dei colori sia insufficiente, considerando le persone affette da daltonismo. Pertanto, un'ulteriore iterazione ci ha portato a caratterizzare maggiormente la \textit{persona} Roberto, esplicitando la sua difficoltà nel riconoscere determinate tonalità.
\noindent
Di seguito riportiamo le \textit{persona} cui siamo giunti, articolandole per tipologia e presentando per ognuna una foto, un elenco puntato delle informazioni salienti richieste e una descrizione discorsiva.

\subsubsection{Protagonisti}
\paragraph{Giornalista testata nazionale}\mbox{}\\
\begin{itemize}
	\item Attitudine:
	\begin{itemize}
        \item esperto con tecnologie informatiche per cui richiede a suo supporto sistemi configurabili secondo le sue esigenze;
        \item è pignolo e esige che gli giungano informazioni in maniera rigorosa e tempestiva;
        \item è oberato di impegni, per cui ha necessità di essere supportato nel suo lavoro da strumenti altamente produttivi e affidabili;
    \end{itemize}
    \item Comportamento: per comprendere i dati sul Covid-19, usa solo dashboard (no articoli, interviste, ecc.) che siano:
	\begin{itemize}
	    \item agevoli nella comprensione;
	    \item affidabili nel funzionamento;
    \end{itemize}
	\item Obiettivi (\textit{end goals}): scrittura rapida di articoli riguardanti l'andamento dell'epidemia, ricavando i dati dalla dashboard del DPC per fornire un'analisi significativa;
	\item Motivazione (\textit{life goals}): fornire informazioni rigorose ai suoi lettori, con particolar attenzione all'analisi dei dati per permettere loro di comprendere approfonditamente la situazione nel contesto, così da prevenire allarmismi o sottostime del fenomeno;
	\item Obiettivi del sistema:
	\begin{itemize}
	    \item sistema di notifiche in caso di aggiornamento dei dati;
	    \item configurabilità interfaccia;
	    \item tracciabilità dei dati (\textit{data lineage}\footnote{La tecnica del \textit{data lineage} consiste nell’identificazione e rappresentazione del ciclo di vita del dato.}).
    \end{itemize}
\end{itemize}

\begin{wrapfigure}{r}{8cm}
    \includegraphics[scale=0.3]{studio-fattibilità/giovanni}
    \caption{\raggedright Foto fantasiosa della persona Giovanni}
\end{wrapfigure}

Giovanni è un uomo di 53 anni di Fiumicino. È felicemente sposato da 24 anni e ha due figlie di 19 e 22 anni.\\ 
Dopo aver completato gli studi presso l'Università La Sapienza di Roma e aver fatto della gavetta presso qualche giornale locale, lavora ormai da 19 anni presso il \textit{Corriere della Sera}, dove ricopre il ruolo di capo redattore. Nell'ultimo anno si è occupato di scrivere articoli per informare i lettori sull'andamento della pandemia dovuta al Covid-19. Ultimamente ha ripreso a lavorare in redazione assieme ad altre 10 persone, ognuno con la propria personale postazione correttamente distanziata dalle altre, in cui è presente un computer fisso con un monitor da 24".
Per scrivere i suoi articoli utilizza un software proprietario, sviluppato per l'azienda, di cui ormai conosce la maggior parte delle funzionalità. Nell'ultimo periodo ha imparato ad utilizzare la dashboard del DPC per raccogliere i dati che inserisce nei suoi articoli.

\paragraph{Giornalista testata locale}\mbox{}\\
\begin{itemize}
    \item Attitudine:
    \begin{itemize}
        \item approccio non rigoroso ai dati, per cui si fida delle dichiarazioni dei media pubblici e privati;
        \item sguardo ristretto alla propria realtà cittadina/regionale, basso interesse per lo scenario nazionale;
    \end{itemize}
	\item Comportamento: 
	\begin{itemize}
	    \item consulta dashboard sul Web, legge articoli e dichiarazioni dei sindaci e del Presidente di regione;
	    \item intervista sindaci e imprenditori locali per ascoltare le loro opinioni o prospettive future;
    \end{itemize}
	\item Obiettivi (\textit{end goals}): raccontare come il Covid-19 abbia influenzato e continui a influenzare la realtà in cui vive e come il territorio reagisce, in termini di:
	\begin{itemize}
        \item nuove sfide per la società (economiche, sociali, ecc.),
        \item criticità che possano coinvolgere i suoi cittadini (chiusura piazze, corsi, centro commerciali, ecc.),
        \item contatti dei presidi sanitari/assistenziali locali e le buone pratiche da adottare;
    \end{itemize}
	\item Motivazione (\textit{life goals}): rendere i suoi concittadini consapevoli dell'evoluzione della pandemia nel loro territorio;
	\item Obiettivi del sistema:
    \begin{itemize}
        \item Possibilità di ricontestualizzare tutti gli elementi comunicativi dal livello nazionale a quello regionale (filtro per regione);
        \item Presentazione delle ordinanze regionali, con focus su quelle della regione di interesse.
    \end{itemize}
\end{itemize}

\begin{wrapfigure}{l}{7cm}
    \includegraphics[scale=0.35]{studio-fattibilità/francesca}
    \caption{Foto fantasiosa della persona Francesca}
\end{wrapfigure}
        
Francesca è una donna di 36 anni di Mestre, laureatasi in Scienze della Comunicazione all'Università di Padova.\\ 
Lavora da sette anni presso la redazione del Gazzettino, in cui cerca di raccontare la propria regione e i suoi cittadini con passione, per fornire il proprio contributo diventato ancora più importante durante questa pandemia.\\ 
Essendo nella sua regione la situazione particolarmente critica, la direzione del giornale ha deciso di far lavorare da remoto e in presenza i suoi dipendenti a settimane alterne, mantenendo invariati i gruppi che lavorano in presenza presso gli uffici in una determinata settimana in modo da poter prevenire delle possibili diffusioni del contagio. Il contatto fisico è ridotto al minimo, tantoché tutte le comunicazioni avvengono rigorosamente tramite lo strumento di messaggistica \textit{Slack}.\\ 
Per via del suo percorso formativo e della passione per il suo territorio, ha a cuore il raccontare ai propri concittadini come i numeri che tutti i giorni sentono in televisione e leggono sui giornali influenzino le loro vite. Per questo motivo, oltre all'analisi dei dati provenienti dalla dashboard del DPC e dai bollettini regionali, intervista i sindaci delle città venete per comprendere le strategie intraprese al fine di fronteggiare l'emergenza e raccoglie le istanze dei piccoli imprenditori locali.

\paragraph{Giornalista testata d'agenzia}\mbox{}\\
\begin{itemize}
	\item Attitudine:
    \begin{itemize}
        \item rigore nella verifica della veridicità delle fonti delle informazioni;
        \item predisposizione a lavorare in maniera rapida, altamente produttiva, disposta ad utilizzare strumenti tecnologici di automatizzazione;
    \end{itemize}
    \item Comportamento: 
    \begin{itemize}
        \item verifica sempre le fonti, ricostruendo il \textit{data lineage};
        \item aggiorna tempestivamente eventuali dati errati/incompleti;
        \item valuta immediatamente le segnalazioni in arrivo da altri presidi di informazione;
    \end{itemize}
    \item Obiettivi (\textit{end goals}): comunicare in maniera puntuale e tempestiva i dati visualizzati nella dashboard del DPC;
    \item Motivazione (\textit{life goals}): informare le altre testate giornalistiche circa i dati veritieri e verificati, così da costituire per loro un punto di riferimento credibile e autorevole;
    \item Obiettivi del sistema:
    \begin{itemize}
        \item notifiche;
        \item link alle fonti.
    \end{itemize}
\end{itemize}

\begin{wrapfigure}{r}{7cm}
    \includegraphics{studio-fattibilità/giulia}
    \caption{Foto fantasiosa della persona Giulia}
\end{wrapfigure}

Giulia è una donna di 35 anni, laureata in Scienze Politiche all'Università di Milano.\\ 
Grande appassionata di tecnologia, si informa su come essa possa semplificarle la vita: infatti, a casa, ha un'assistente vocale e numerose luci smart, controllabili tramite il suo smartphone.
Convive da otto anni con il suo fidanzato in un piccolo appartamento nel centro di Milano, a 10 minuti a piedi dall'ufficio in cui lavora. Condivide l'ufficio assieme ad altre 4 persone, lavorando sul portatile datole dall'agenzia di stampa con il quale può lavorare anche da casa. L'agenzia richiede che il lavoro sia svolto principalmente in smart working finché la situazione epidemiologica non migliora: prevede una deroga per stretta necessità per cui ci si può recare in ufficio.
Il suo ruolo nell'agenzia è quello di ``fact-checking", ossia raccogliere i dati relativi ad un certo fenomeno e verificarne la veridicità, in modo che non vengano pubblicati dati errati o incompleti.\\ 
Nell'ultimo periodo i dati che maggiormente ha studiato sono quelli relativi all'epidemia Covid-19. La sua fonte principale per questo tema è la dashboard del DPC ma fa riferimento in maniera rigorosa e sistematica anche alle fonti riferite in essa. Per il suo lavoro si trova spesso a ricostruire il ciclo di vita di ogni dato così da verificarne la veridicità.\\ 
Per rimanere in comunicazione con i colleghi utilizza chat di gruppo, sistemi di collaborazione e file sharing, riuscendo a destreggiarsi con agevolezza tra le funzionalità che essi offrono. 

\subsubsection{Persona secondarie}
\paragraph{Opinionista/blogger}\mbox{}\\
\begin{itemize}
    \item Attitudine:
    \begin{itemize}
        \item mentalità aperta;
        \item volontà di andare a fondo nelle notizie, raccogliendo pareri disparati;
    \end{itemize}
    \item Comportamento: 
    \begin{itemize}
        \item lettura mattutina delle informazioni che monitora nel resto della giornata mediante RSS;
        \item interazione con i suoi follower, ascolto delle loro opinioni e raccolta dei loro interessi, per crearvi nuovi contenuti più attraenti per la sua \textit{community};
    \end{itemize}
    \item Obiettivi (\textit{end goals}): creare contenuti relativi all'epidemia, basando le sue argomentazioni sui dati della dashboard del DPC;
    \item Motivazione (\textit{life goals}): stimolare una libera discussione tra i suoi follower (lettori, ascoltatori, ecc.) sui temi caldi del momento;
    \item Obiettivi del sistema:
    \begin{itemize}
        \item possibilità di esportare elementi grafici in formato immagine o widget HTML da incorporare;
        \item possibilità di selezionare solo alcuni elementi grafici per comporli in una mini dashboard da incorporare; 
        \item ogni grafico deve utilizzare modalità multiple per veicolare informazioni (mouse over per esplicitare contenuto informativo di componenti grafiche, icone, ecc.). 
    \end{itemize}
\end{itemize}

\begin{wrapfigure}{l}{7cm}
    \includegraphics[scale=0.68]{studio-fattibilità/francesco}
    \caption{Foto fantasiosa della persona Roberto}
\end{wrapfigure}

Roberto è un uomo di 30 anni, che dopo aver terminato gli studi presso il liceo scientifico ha deciso di viaggiare in giro per l'Europa per un anno. Roberto soffre di deuteranopia (tipologia di daltonismo). Durante quest'anno ha conosciuto la sua attuale ragazza ed ora vivono assieme in un paesino vicino a Bellinzona. Roberto, grazie alle sue conoscenze informatiche ha creato un blog e ha aperto un canale \textit{YouTube}. I contenuti che produce trattano di attualità ed è seguito da un vasto pubblico di ragazzi tra i 16 e i 25 anni. Nel suo appartamento ha una stanza interamente dedicata alla realizzazione dei video che poi pubblica sul suo canale \textit{YouTube}, in questa stanza trova anche spazio una scrivania con 2 monitor da 27'' e un computer fisso di ultima generazione. Per aggiornare il suo blog utilizza \textit{WordPress} che ormai non ha più segreti per lui.\\ 
Da Marzo 2019 ha iniziato una rubrica settimanale nella quale racconta l'evoluzione della situazione epidemiologica in Italia e nel mondo, che segue con rigore scientifico, al fine di riuscire a spiegare in maniera semplice e accurata i difficili dati che i giornali divulgano. Utilizza diverse fonti per raccogliere i dati e svariati strumenti per analizzarli. 
\pagebreak
\subsubsection{Persona addizionali}
\paragraph{Cittadino esperto}\mbox{}\\
	\begin{itemize}
	\item Attitudine:
    \begin{itemize}
        \item vuole maturare una propria opinione sugli argomenti di attualità;
        \item vuole contribuire ad un'informazione sana e corretta dei suoi familiari;
        \item ragionevole e responsabile, incline al rispetto delle istituzioni, fiducioso nella scienza;
    \end{itemize}
	\item Comportamento: 
    \begin{itemize}
        \item connessione alla dashboard sporadica, solo quando ha necessità di informarsi o ha ritagli di tempo libero;
        \item condivide e discute circa le informazioni acquisite con colleghi e amici;
    \end{itemize}
    \item Obiettivi (\textit{end goals}): integrare quanto letto presso articoli di giornali di varia natura;
    \item Motivazione (\textit{life goals}): acquisire maggiore consapevolezza e conoscenza dei fenomeni che impattano la realtà in cui vive, nonché la capacità di discernere l'essenziale dal \textit{mare magnum} di informazioni che, continuamente, lo investono;
    \item Obiettivi del sistema:
    \begin{itemize}
        \item filtro per regione;
        \item condivisione tramite mail/social network di una certa configurazione della dashboard per sostenere le sue riflessioni;
        \item estetica accattivante e grafici interattivi.
    \end{itemize}
\end{itemize}

\begin{wrapfigure}{r}{8cm}
    \includegraphics[scale=1.5]{studio-fattibilità/fabio}
    \caption{Foto fantasiosa della persona Fabio}
\end{wrapfigure}

Fabio è un uomo di 62 anni del Pescarese, la cui quotidianità è illuminata dalla sua famiglia e dal suo lavoro come docente universitario di Informatica presso l'Università degli Studi G. D'Annunzio (Chieti-Pescara).\\ 
È sposato da 25 anni con Claudia, infermiera presso l'Ospedale S. Spirito di Pescara ed è padre di un'adolescente di 17 anni e un giovane di 23 anni: una delle abitudini più preziose della famiglia di Fabio è la condivisione della cena, durante la quale si parla delle attività di ognuno e si intavolano riflessioni circa le ultime notizie dall'Italia e dal mondo.\\
Negli anni, Fabio ha cercato di trasmettere ai figli il profondo rispetto che riconosce alle istituzioni, oltre alla fiducia che ripone nella scienza. Inoltre, è molto sensibile al dilagare delle fake news nei diversi ambiti: crede che siano responsabili di una cattiva informazione delle persone, e quindi di una loro consapevolezza distorta della realtà.\\ 
Queste sue considerazioni emergono non solo in famiglia ma anche nei momenti di svago, ad esempio durante le chiacchierate con gli amici.\\ 
Ogni giorno, Fabio cerca di informarsi tramite canali considerati attendibili e autorevoli; negli ultimi mesi imperversati dalla pandemia Covid-19, ha esteso il suo briefing mattutino anche alla dashboard del DPC: grazie alle sue competenze informatiche, riesce a consultarla in maniera agevole. In particolare, vi si rivolge quando gli articoli di giornale che legge si rivelano confusionari o non allineati: la dashboard, presentando dati in maniera immediata e chiara, permette a Fabio di dirimere i dubbi e acquisire l'essenziale; di frequente, estrae elementi grafici dalla dashboard per poi condividerli con i contatti dei suoi social, a corredo di proprie riflessioni.

\subsubsection{Persona non-utenti, negative}
\paragraph{Utente che utilizza lo smartphone per accedere alla dashboard}\mbox{}\\
\vspace{-5mm}
\begin{itemize}
	\item Attitudine:
    \begin{itemize}
        \item vita frenetica;
        \item non dedica un momento della propria routine all'informarsi in maniera sistematica dal computer;
    \end{itemize}
    \item Comportamento: si informa tramite le rassegne stampa;
    \item Obiettivi (\textit{end goals}): fare ipotesi sulle successive evoluzioni del Covid-19 e sui futuri provvedimenti governativi;
    \item Motivazione (\textit{life goals}): informarsi sommariamente su quel che sta accadendo;
    \item Obiettivi del sistema:
    \begin{itemize}
        \item piccolo testo che informa l'utente che la fruizione consigliata è in modalità desktop;
        \item ridurre le funzionalità al minimo per garantire accesso veloce e chiaro su schermi piccoli.
    \end{itemize}
\end{itemize}

\begin{wrapfigure}{l}{7cm}
    \includegraphics[scale=0.2]{studio-fattibilità/christian}
    \caption{Foto fantasiosa della persona Christian}
\end{wrapfigure}

Christian è un uomo di 34 anni che lavora come fattorino per un'agenzia di logistica. Ha un appartamento a Torino vicino a una fermata della metropolitana e per raggiungere la propria sede di lavoro deve necessariamente prenderla quotidianamente. La sua giornata di lavoro inizia con la raccolta dei pacchi che consegnerà con uno dei furgoni dell'azienda. Le restrizioni imposte alla mobilità delle persone dai vari DPCM ha aumentato notevolmente i clienti dei negozi online e di conseguenza il lavoro di Christian ne ha risentito parecchio. Al ritorno a casa, la sera, non gli avanza molto tempo libero in quanto vive da solo e deve provvedere anche alle faccende di casa. Inoltre, la stanchezza lo porta a desiderare di andare a dormire quanto prima, magari dopo essersi un po' riposato sul suo divano.\\
Nonostante il poco tempo a disposizione, vuole essere informato su ciò che avviene intorno a sé. Al mattino, prendendo la metropolitana, e in pausa pranzo, ascolta le rassegne stampa e legge alcuni giornali che ritiene affidabili, tra cui il \textit{Corriere della Sera}, attraverso il suo smartphone con uno schermo da 6''. Se gli capita, legge anche il bollettino che l'\textit{ANSA} ha pubblicato il giorno prima e che riporta i dati aggiornati sull'andamento dell'epidemia.


