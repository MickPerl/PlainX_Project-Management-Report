\subsection{Persona}
Abbiamo individuato e definito dettagliatamente sei \textit{persona}\footnote{Con il termine \textit{persona} si intende un utente sintetico ideato in modo che incarni tutte gli aspetti (intenzioni, obiettivi, attitudini, comportamenti, abitudini...) differenti che si vogliono supportare nell'ambito del design di un prodotto.}, ciascuna raffigurante un profilo interessante e peculiare.\\
In dettaglio, in un primo momento, abbiamo individuato 3 \textit{persona}: il protagonista era il giornalista, il blogger era il personaggio secondario, il cittadino esperto l'addizionale. Nel definire il protagonista, tuttavia, ci siamo resi conto dell'eccessivo mole di attività che avrebbe potuto compiere per mezzo della dashboard, per cui abbiamo ritenuto opportuno articolarlo in distinte \textit{persona}: il giornalista di una testata nazionale, il giornalista di una testata locale, il giornalista d'agenzia. Crediamo che tale cambiamento ci possa portare ad una maggior grado di dettaglio nella definizione delle \textit{persona} stesse, così da esser maggiormente guidati nella successiva progettazione dell'interfaccia.
\noindent
In realtà, prima di giungere a quest'ultima articolazione, ne avevamo prevista una differente, in cui era presente anche il giornalista di testate esclusivamente online: tuttavia, nella sua definizione ci siamo accorti di numerose intersezioni e differenze poco significative con la \textit{persona} del blogger e del giornalista di testate nazionali o locali. \\ 
Modifiche successive sono intervenute anche sulla \textit{persona} ``cittadino": dopo un'attenta riflessione sulle sue potenziali peculiarità, abbiamo individuato la possibilità di scinderla in due \textit{persona} distinte, quella del ``cittadino esperto", come \textit{persona} addizionale e quella dell'``utente che utilizza smartphone", come \textit{persona} negativa.\\
Dall'analisi più approfondita delle dashboard preesistenti, ci siamo resi conto della presenza importante di colori come strumento per distinguere gli elementi grafici: essendo per tali applicazioni web di assoluta rilevanza il ``colpo d'occhio", crediamo che affidarsi solamente alla capacità distintiva dei colori sia insufficiente, considerando le persone affette da daltonismo. Pertanto, un'ulteriore iterazione ci ha portato a caratterizzare maggiormente la \textit{persona} Roberto, esplicitando la sua difficoltà nel riconoscere determinate tonalità.
\noindent
Di seguito riportiamo le \textit{persona} cui siamo giunti, articolandole per tipologia e presentando per ognuna una foto, un elenco puntato delle informazioni salienti richieste e una descrizione discorsiva.
\newpage
\subsubsection{Protagonisti}

\subfile{persona/giornalista-testata-nazionale}

\subfile{persona/giornalista-testata-locale}

\subfile{persona/giornalista-testata-agenzia}

\subsubsection{Persona secondarie}

\subfile{persona/blogger}

\subsubsection{Persona addizionali}

\subfile{persona/cittadino-esperto}

\subsubsection{Persona non-utenti, negative}

\subfile{persona/utente-dispositivo-mobile}