\subsection{Contesto d'uso}
\subsubsection{Utenti del sistema}
Da una prima analisi delle dashboard esistenti sul \textit{Coronavirus}, inclusa quella oggetto della riprogettazione, siamo giunti alla consapevolezza che fosse necessario un background scientifico per consultarle agevolmente e trarne informazioni.\\
A seguito di una valutazione più approfondita, abbiamo ravvisato innumerevoli criticità in termini di presentazione ed elaborazione dei dati: queste criticità sono le cause principali della scarsa usabilità e della misera esperienza utente che  fanno preferire ai giornalisti i bollettini e i lanci di agenzia.\\
Il nostro processo iterativo, a questo punto, si è spinto a considerare possibile una riprogettazione della Dashboard ufficiale, ossia quella pubblicata dal \textit{Dipartimento della Protezione Civile}, indirizzando il nostro design a giornalisti con solo background umanistico; tuttavia, ci siamo resi conto che questa impostazione avrebbe reso la dashboard certamente chiara per chiunque, ma forse limitata nelle funzionalità.\\ 
Ancora una volta, pertanto, abbiamo cambiato la nostra platea di riferimento, decidendo di rivolgerci tanto a giornalisti con background umanistico quanto a quelli dalla formazione scientifica: l'inesperienza dei primi ci imporrà la progettazione di un sistema all'insegna dell'usabilità, la perizia dei secondi ci porterà a implementare funzionalità più avanzate.

In altri termini, il segmento di utenza cui è indirizzata la nostra riprogettazione considera \emph{i giornalisti che pubblicano articoli sull'andamento di vari aspetti (quantitativi) della pandemia Covid-19 (contagi giornalieri, decessi, guariti, tamponi, occupazione terapie intensive, ricoveri...): questi possono vantare una formazione umanistica (Liceo classico, facoltà di area umanistica) ovvero scientifica (Liceo scientifico, Scienze matematiche, fisiche e naturali (in sigla MM.FF.NN.)).}

\subsubsection{Elenco delle attività}
\begin{itemize}
    \item Comprendere l'andamento della curva epidemiologica sulla base del rapporto tra i nuovi positivi e il numero di tamponi effettuati;
    \item Monitorare l'occupazione delle strutture sanitarie (ricoveri con sintomi, ricoveri in terapia intensiva);
    \item Monitorare l'andamento del tasso di mortalità;
    \item Come i numeri dell'epidemia si distribuiscono sulla popolazione (età, malattie pregresse, impiego, …);
    \item Confronto dell'andamento dell'epidemia tra due o più regioni;
    \item Confronto dell'andamento dell'epidemia tra periodi temporali diversi;
\end{itemize}

L'individuazione delle attività sopra elencate deriva dalla lettura di numerosi articoli sul tema e dai risultati del questionario sottosposto ai giornalisti, nonché dall'intervista alla giornalista Roberta, di \textit{SkyTg24}: rivolgendoci ai diretti interessati, abbiamo potuto scoprire attività che ignoravamo precedentemente, riuscendo ad ottenere una panoramica completa del lavoro di chi crea informazione sul tema.

\subsubsection{Vincoli tecnici e d'ambiente}
A partire dalle risposte raccolte mediante il questionario sottoposto ai giornalisti e dai feedback preziosi tratti dall'intervista alla giornalista Roberta di cui sopra, abbiamo individuato i seguenti vincoli di natura tecnica o imposti dall'ambiente il cui il nostro utente di riferimento opera:
\begin{itemize}
    \item Il layout della dashboard deve essere adatto a schermi in modalità orizzontale dai 13 pollici in sù;
    \begin{itemize}
        \item Il giornalista è in redazione o in smart working, ha dinanzi a sé uno schermo sufficientemente grande per lavorare agevolmente oppure un insieme di schermi grandi a parete  per la visione comoda dell'intero ufficio: conseguentemente, il sistema è pensato principalmente per essere fruito su schermi in modalità panoramica (orizzontale) dai 13 pollici in su. Un fallback per dispositivi mobili può essere disponibile, ma non è la priorità.
    \end{itemize}
    \item Notifica presenza di nuovi dati
    \begin{itemize}
        \item Il giornalista sta lavorando, è concentrato e non vuole uscire dal flow raggiunto: essendo a conoscenza del fatto che i dati giornalieri sulla Covid-19 sono pubblicati attorno alle ore 17:30±30min, potrebbe aver aperto la dashboard in precedenza e vuole essere notificato non appena i nuovi dati sono disponibili.
    \end{itemize}
	\item Comunicazione delle informazioni in maniera diretta e chiara, senza fronzoli
	\begin{itemize}
        \item Ricevuta la notifica vuole, con rapidi sguardi, ottenere le informazioni di interesse, in un formato leggibile, chiaro e comprensibile, così da avere immediatamente il materiale necessario per procedere alla stesura del suo articolo in materia.
        \item Sistema militare, dritto al punto!
        \item Esempio da non emulare: articoli di blog sull'uscita di una stagione successiva di una serie tv (la risposta che si vuole ottenere è un sì/no, ma l'articolo inserisce trama, immagini, informazioni sugli attori, pubblicità e infine un paragrafo inutile in cui probabilmente scrivono "non si sa")
    \end{itemize}
	\item Il contenuto informativo visualizzato deve poter filtrato su più criteri
	\begin{itemize}
        \item Nell'ottica di una comprensione più profonda/ampia dell'argomento, il giornalista vuole interagire con i vari elementi della dashboard così da personalizzarne il contenuto informativo e ottenere informazioni aderenti ai suoi bisogni (filtro su regioni, finestre temporali specifiche…).
    \end{itemize}
	\item Il sistema deve essere usabile, ma allo stesso tempo fornire funzionalità avanzate
	\begin{itemize}
        \item sulla base di quanto emerso dal questionario, i giornalisti hanno tendenzialmente un'alta familiarità con la tecnologie, pertanto si attendono un sistema che permetta loro di svolgere task complessi, che utilizzi terminologia specifica e che non li tratti da novizi (ben venga, un tasto di help che espliciti su richiesta la semantica di un elemento, ma evitare suggerimenti permanenti o tutorial iniziali).
    \end{itemize}
	\item Devono essere presentati dati relativi e/o aggregati
	\begin{itemize}
        \item Uno dei principali motivi che, a quanto appreso dal questionario, porta i giornalisti a non consultare la dashboard della Protezione civile è la presenza, pressocchè esclusiva, di dati assoluti, non relativizzati ai contesti in cui emergono (nr positivi su 100.000 abitanti, nr positivi su nr tamponi molecolari effettuati…);
    \end{itemize}
\end{itemize}
