\subsection{Contesto d'uso}
\subsubsection{Utenti del sistema}
Da una prima analisi delle dashboard esistenti sul \textit{Coronavirus}, inclusa quella oggetto della riprogettazione, siamo giunti alla consapevolezza che fosse necessario un background scientifico per consultarle agevolmente e trarne informazioni.\\
A seguito di una valutazione più approfondita, abbiamo ravvisato innumerevoli criticità in termini di presentazione ed elaborazione dei dati: queste criticità sono le cause principali della scarsa usabilità e della misera esperienza utente che  fanno preferire ai giornalisti i bollettini e i lanci di agenzia.\\
Il nostro processo iterativo, a questo punto, si è spinto a considerare possibile una riprogettazione della Dashboard ufficiale, ossia quella pubblicata dal \textit{Dipartimento della Protezione Civile}, indirizzando il nostro design a giornalisti con solo background umanistico; tuttavia, ci siamo resi conto che questa impostazione avrebbe reso la dashboard certamente chiara per chiunque, ma forse limitata nelle funzionalità.\\ 
Ancora una volta, pertanto, abbiamo cambiato la nostra platea di riferimento, decidendo di rivolgerci tanto a giornalisti con background umanistico quanto a quelli dalla formazione scientifica: l'inesperienza dei primi ci imporrà la progettazione di un sistema all'insegna dell'usabilità, la perizia dei secondi ci porterà a implementare funzionalità più avanzate.

In altri termini, il segmento di utenza cui è indirizzata la nostra riprogettazione considera \emph{i giornalisti che pubblicano articoli sull'andamento di vari aspetti (quantitativi) della pandemia Covid-19 (contagi giornalieri, decessi, guariti, tamponi, occupazione terapie intensive, ricoveri...): questi possono vantare una formazione umanistica (Liceo classico, facoltà di area umanistica) ovvero scientifica (Liceo scientifico, Scienze matematiche, fisiche e naturali (in sigla MM.FF.NN.)).}

\subsubsection{Elenco delle attività}
\begin{itemize}
    \item Comprendere l'andamento della curva epidemiologica sulla base del rapporto tra i nuovi positivi e il numero di tamponi effettuati;
    \item Monitorare l'occupazione delle strutture sanitarie (ricoveri con sintomi, ricoveri in terapia intensiva);
    \item Monitorare l'andamento del tasso di mortalità;
    \item Come i numeri dell'epidemia si distribuiscono sulla popolazione (età, malattie pregresse, impiego, …);
    \item Confronto dell'andamento dell'epidemia tra due o più regioni;
    \item Confronto dell'andamento dell'epidemia tra periodi temporali diversi;
\end{itemize}

L'individuazione delle attività sopra elencate deriva dalla lettura di numerosi articoli sul tema e dai risultati del questionario sottosposto ai giornalisti, nonché dall'intervista alla giornalista Roberta, di \textit{SkyTg24}: rivolgendoci ai diretti interessati, abbiamo potuto scoprire attività che ignoravamo precedentemente, riuscendo ad ottenere una panoramica completa del lavoro di chi crea informazione sul tema.


