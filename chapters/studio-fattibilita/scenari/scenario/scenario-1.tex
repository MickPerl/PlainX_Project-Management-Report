\subsubsection{Scenario 1: Analisi dei dati quotidiani sull'andamento dell'epidemia Covid-19 in Italia}
Dopo una lunga giornata di lavoro, alle 17, il giornalista Maurizio che si occupa degli articoli sull'andamento dell'epidemia Covid-19 in Italia, si collega alla dashboard del DPC dal suo portatile da 15".\\
Tempo stimato per:
\begin{enumerate}
    \item l'apertura e il caricamento della dashboard: \textit{20 s}.
\end{enumerate}

\noindent
Vuole scrivere l'articolo del giorno nel quale comunica ai lettori i nuovi valori delle metriche principali, tra cui il numero di nuovi positivi, il numero di deceduti, il numero di nuovi ingressi in terapia intensiva, il rapporto tra i nuovi positivi e i tamponi effettuati (tasso di positività). Dal momento che queste metriche sono consultate giornalmente, la dashboard le presenta in maniera immediata ed evidente: Maurizio le coglie con rapide occhiate, senza dover effettuare operazioni o calcoli particolari e, tantomeno, dovendosi rivolgere ad altri strumenti o fonti.\\
Tempo richiesto per:
\begin{enumerate}
    \item la visione dei dati: \textit{5 s};
    \item l'analisi dei dati per la successiva scrittura dell'articolo: \textit{50 s}.
\end{enumerate}

\noindent
Inoltre, Maurizio intende studiare l'andamento delle metriche in rapporto ai giorni precedenti, per cui chiede alla dashboard di restituirgli, oltre ai dati relativi alla giornata corrente, anche quelli del/dei periodo/i temporale/i di suo interesse: la dashboard prontamente presenta quanto richiesto in un formato che favorisce agili confronti e permette di trarre conclusioni di valore.\\
Tempo richiesto per:
\begin{enumerate}
    \item la visualizzazione dei dati, la selezione del periodo e il caricamento dei dati: \textit{8 s};
    \item l'analisi dei dati \textit{100 s}.
\end{enumerate}

\noindent
Ancora, Maurizio richiede alla dashboard di visualizzare i valori di certe metriche a livello regionale: in dettaglio, la dashboard soddisfa la richiesta e permette di ordinare le regioni in base al valore che assumono per una o più delle metriche indicate, così da fornire a Maurizio una panoramica sulla loro situazione.\\
Tempo richiesto per:
\begin{enumerate}
    \item la selezione delle metriche e l'ordinamento: \textit{40 s};
    \item l'analisi confrontando ordinamenti per diverse metriche: \textit{130 s}.
\end{enumerate}

\noindent
Maurizio nota degli \textit{outlier}:  leggendo le note del giorno, appura che certe regioni hanno avuto problemi con la comunicazione dei dati.\\
Tempo richiesto per:
\begin{enumerate}
    \item individuare la nota di riferimento, verificare la presenza dell'outlier: \textit{35 s};
    \item l'analisi delle possibili implicazioni di questi outlier sull'andamento delle metriche coinvolte: \textit{115 s}.
\end{enumerate}

\noindent
Attività svolte: \hyperref[itm:1]{\textit{1}}, \hyperref[itm:2]{\textit{2}}, \hyperref[itm:4]{\textit{4}}.\\ 
Tempo totale: \textit{8' 23''}. 