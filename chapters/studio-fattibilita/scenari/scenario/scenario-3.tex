\subsection{Scenario 3: Analisi sull'andamento del tasso di letalità e sulla distribuzione dei numeri dell'epidemia relativamente alle due settimane precedenti}
Francesco, ogni due settimane, deve analizzare il tasso di letalità e la distribuzione dei numeri dell’epidemia sulla popolazione per scrivere un articolo di giornale che sarà pubblicato da un importante editore italiano. Durante il periodo che precede la pubblicazione, accede diverse volte alla dashboard per analizzare i dati di suo interesse.\\
Tempo richiesto per:
\begin{enumerate}
    \item l'apertura e il caricamento della dashboard: \textit{20 s}.
\end{enumerate}

\noindent
Quindi, Francesco indica il periodo che vuole analizzare e i dati anagrafici (genere e fascia d'età) su cui vuole analizzare la distribuzione dei dati epidemiologici.
I dati ottenuti gli permettono di scrivere un articolo preciso e approfondito. Inoltre, riesce a dare informazioni più dettagliate ai suoi lettori su chi sono le persone dietro i numeri dell'epidemia: la loro età, il loro genere, il loro impiego (ovvero se sono operatori sanitari).\\
Tempo richiesto per:
\begin{enumerate}
    \item la selezione del periodo, dei dati anagrafici di interesse e il caricamento: \textit{12 s};
    \item l'analisi dei dati \textit{125 s}.
\end{enumerate}

\noindent
Esplorando i dati ed ordinandoli per numero di decessi, ravvisa una possibile correlazione tra la mortalità e l'età: nell'intenzione di validare questa sua ipotesi, richiede alla dashboard che gli presenti, ad una granularità maggiore e con aggregazioni più significative, i valori relativamente a queste due variabili. 
Tempo richiesto per:
\begin{enumerate}
    \item la scoperta di alcune correlazioni, l'impostazione della dashboard per visualizzare i dati di suo interesse e il caricamento: \textit{210 s};
    \item l'analisi dei dati \textit{140 s}.
\end{enumerate}

\noindent
Attività svolte: \hyperref[itm:3]{\textit{3}}, \hyperref[itm:4]{\textit{4}}.\\ 
Tempo totale: \textit{8' 27''}.