\subsubsection[Scenario 4: Confronto dell'andamento dell'epidemia nelle regioni Emilia Romagna, Veneto e Molise nelle precedenti due settimane]{Scenario 4: Confronto dell'andamento dell'epidemia nelle regioni Emilia Romagna, Veneto e Molise\footnote{Anche in questo caso, come per Scenario 2, le regioni sono tre e specifiche, ma possono essere di meno o di più e anche essere altre regioni italiane.} nelle precedenti due settimane}
Matteo, il giornalista che nei suoi articoli confronta l'andamento della pandemia in regioni diverse, alle ore 17 si collega alla dashboard del DPC dalla postazione di lavoro adibita in casa sua, ove dispone di un comodo schermo curvo da 32".\\
Tempo stimato per:
\begin{enumerate}
    \item l'apertura e il caricamento della dashboard: \textit{20 s}.
\end{enumerate}

\noindent
In dettaglio, vuole scrivere l'articolo a cadenza bisettimanale in cui confronta i dati della pandemia nelle  regioni che si sono distinte sullo scenario nazionale per l'evoluzione di una o più metriche: l'obiettivo di Matteo è quello di far emergere tratti comuni e aspetti distintivi che permettano di trarre conclusioni significative per comprendere al meglio il livello di gravità e rischio di ogni regione.

Vuole recuperare le varie metriche pubblicate negli articoli giornalieri, ma sintetizzate con riferimento al periodo delle due settimane appena trascorse (il rapporto medio tra nuovi positivi e tamponi nel periodo, la media del numero di ospedalizzazioni giornaliere, decessi cumulati, ecc.).
Precisamente, per l'articolo corrente, intende concentrarsi sulle regioni Emilia Romagna, Veneto e Molise.\\
Tempo richiesto per:
\begin{enumerate}
    \item impostare la dashboard sul periodo e sulle regioni di interesse: \textit{15 s}.
\end{enumerate}

\noindent
Procede quindi col richiedere alla dashboard di visualizzare contemporaneamente i dati della pandemia relativi alle tre regioni di interesse, in un formato chiaro e che gli permetta un agile confronto: la dashboard, immediatamente, gli restituisce calcoli automatici, come rapporti e differenze, tramite cui può comprendere, con dati alla mano, le reali differenze o somiglianze.\\
Tempo richiesto per:
\begin{enumerate}
    \item la selezione dei dati di interesse, delle metriche aggregate e il caricamento: \textit{20 s};
    \item l'analisi dei dati \textit{135 s}.
\end{enumerate}

\noindent
A Matteo pare di notare degli \textit{outlier} ma analizzando le note del giorno si accorge che in realtà i dati sono stati comunicati correttamente e che quindi non vi sono errori. Matteo può quindi procedere alla redazione del suo articolo.\\
Tempo richiesto per:
\begin{enumerate}
    \item cercare le note e non trovarle: \textit{30 s}.
\end{enumerate}

\noindent
Attività svolte: \hyperref[itm:5]{\textit{5}}.\\  
Tempo totale: \textit{3' 40''}. 