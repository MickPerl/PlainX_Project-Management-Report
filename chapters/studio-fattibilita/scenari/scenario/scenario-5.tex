\subsection{Scenario 5: Confronto dell'andamento dell'epidemia in Italia tra periodi temporali diversi}
Dopo una settimana in cui Maurizio, il giornalista che si occupa degli articoli sull'andamento dell'epidemia Covid-19 in Italia, ha raccontato ai suoi lettori come la situazione si stava evolvendo giorno per giorno e dopo aver redatto anche l'articolo quotidiano, resta collegato sulla dashboard del DPC col suo portatile da 15" per scrivere un ulteriore articolo, questa volta sul confronto fra il mese corrente e il precedente.\\
Tempo richiesto per:
\begin{enumerate}
    \item l'apertura e il caricamento: \textit{0 s}.
\end{enumerate}

\noindent
Maurizio inserisce i periodi temporali che vuole confrontare e la dashboard gli presenta i dati che sono stati raccolti in un formato che agevola confronti diretti e indiretti. In particolare, è interessato all'andamento del tasso di positività, del tasso di letalità, delle ospedalizzazioni e dell'occupazione delle terapie intensive.\\
Tempo richiesto per:
\begin{enumerate}
    \item la selezione dei periodi, la raccolta dei dati e l'impostazione della visualizzazione per il confronto: \textit{35 s};
    \item l'analisi dei dati \textit{210 s}.
\end{enumerate}

\noindent
Analizzando i dati dei mesi di settembre, ottobre e novembre nota che vi è una significativa diminuzione del numero di contagi con l'inasprimento delle misure restrittive soprattutto a fine novembre, esattamente due settimane dopo l'entrata in vigore del DPCM più recente.\\
Tempo richiesto per:
\begin{enumerate}
    \item individuare la sezione con i provvedimenti normativi approvati in materia di contrasto all'epidemia, espansione e lettura dei punti salienti dei provvedimenti: \textit{100 s};
    \item l'analisi delle possibili correlazioni di questi provvedimenti sui dati: \textit{140 s}.
\end{enumerate}

\noindent
Attività svolte: \hyperref[itm:6]{\textit{6}}.\\ 
Tempo totale: \textit{8' 5''}. 