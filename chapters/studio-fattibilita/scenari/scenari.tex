\section{Scenari}
\label{s:scenari}
In prima battuta, per ogni scenario, abbiamo descritto solo l'\textit{happy path}\footnote{Nel contesto del software o della modellazione delle informazioni, un \textit{happy path} è uno scenario predefinito che non presenta condizioni eccezionali o di errore.}: ritornando a riflettere sulla loro prima versione, abbiamo ritenuto necessario dettagliare anche situazioni anomale, ad esempio la presenza di \textit{outlier}\footnote{Gli \textit{outlier} sono valori anomali, considerevolmente differenti dalla maggior parte dei dati dell'insieme a cui appartengono} o di aggiornamenti tardivi dei dati, al fine di giungere a scenari quanto più vicini alla realtà.

\subfile{scenario/scenario-1}

\subfile{scenario/scenario-2}

\subfile{scenario/scenario-3}

\subfile{scenario/scenario-4}

\subfile{scenario/scenario-5}
