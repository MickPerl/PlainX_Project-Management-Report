\subsection{Scenari}
\subsubsection*{Scenario 1: Analisi dei dati quotidiani sull'andamento dell'epidemia Covid-19 in Italia}
Dopo una lunga giornata di lavoro, alle 17 Maurizio, il giornalista che si occupa degli articoli sull'andamento dell'epidemia Covid-19 in Italia, si collega alla dashboard della Protezione Civile dal suo portatile 15".
(Tempo stimato per l'apertura e il caricamento della dashboard: 20 s)

Vuole scrivere l'articolo del giorno nel quale comunica ai lettori l'andamento di varie metriche, per esempio il rapporto tra nuovi positivi e tamponi effettuati quel giorno oppure le ospedalizzazioni giornaliere. Essendo le metriche fondamentali per scrivere l'articolo le trova immediatamente, senza dover effettuare operazioni, calcoli e senza aver bisogno di utilizzare altri strumenti o fonti.  
(Tempo richiesto per la visione dei dati: 5 s - Tempo richiesto per l'analisi dei dati per la successiva scrittura dell'articolo: 50 s)

Per studiarne l'andamento rispetto ai giorni precedenti, Maurizio chiede alla dashboard che gli siano restituiti non solo i dati relativi alla giornata corrente ma anche quelli dell'intero periodo temporale di interesse: la dashboard prontamente presenta quanto richiesto in modo che il giornalista possa comprendere l'evoluzione nel tempo delle varie metriche. (Tempo richiesto per la visualizzazione dei dati, la selezione del periodo e il carimento: 8 s - Tempo richiesto per l'analisi dei dati 100 s)

Inoltre, Maurizio, richiede alla dashboard di visualizzare alcune metriche di interesse per ogni regione, con la possibilità di ordinarle, in modo da capire la situazione e la gravità in ciascuna di esse. (Tempo richiesto per la selezione delle metriche e l'ordinamento: 40 s - Tempo richiesto per l'analisi confrontando diversi ordinamenti e diverse metriche: 130 s)

Maurizio nota degli outlier (valori fuori linea rispetto a quelli precedenti):  analizzando le note del giorno trova scritto che certe regioni hanno avuto problemi con la comunicazione di alcune metriche. (Tempo richiesto per individuare la nota di riferimento, verificare la presenza dell'outlier: 35 s - Tempo richiesto per l'analisi delle possibili implicazioni di questo outlier sui dati: 115 s)

Task svolti: 1,2
Tempo totale: 8' 23''

\subsubsection*{Scenario 2: Analisi dei dati quotidiani sull'andamento dell'epidemia Covid-19 nell'Emilia Romagna}
Alle 17 Giulia, la giornalista che si occupa degli articoli sull'andamento dell'epidemia Covid-19 in Emilia Romagna, si collega alla dashboard della Protezione Civile dal suo portatile 13". Al momento, Giulia, è in smart-working. (Tempo stimato per l'apertura e il caricamento: 20 s)

Vuole scrivere l'articolo del giorno nel quale comunica ai lettori l'andamento di varie metriche, per esempio il rapporto tra nuovi positivi e tamponi effettuati quel giorno oppure le ospedalizzazioni giornaliere. Essendo di suo interesse solo la regione Emilia Romagna richiede alla dashboard di visualizzare solo i dati per quella regione. Le metriche di interesse per l'articolo vengono subito presentate senza dover effettuare operazioni o calcoli e senza aver bisogno di utilizzare altri strumenti o fonti.
(Tempo richiesto per la selezione della regione e la visione dei dati: 10 s - Tempo richiesto per l'analisi dei dati per la successiva scrittura dell'articolo: 50 s)

Per studiarne l'andamento rispetto ai giorni precedenti, Giulia chiede alla dashboard che le siano restituite non solo i dati relativi alla giornata corrente ma anche quelli dell'intero periodo temporale di interesse: la dashboard prontamente presenta quanto richiesto in modo che la giornalista possa comprendere l'evoluzione nel tempo delle varie metriche. (Tempo richiesto per la visualizzazione dei dati, la selezione del periodo e il carimento: 8 s - Tempo richiesto per l'analisi dei dati 100 s)

Giulia potrebbe notare degli outlier (valori fuori linea rispetto a quelli precedenti) e quindi analizzando le note del giorno legge che la regione Emilia Romagna ha avuto problemi con la comunicazione degli ultimi dati. (Tempo richiesto per individuare la nota di riferimento, verificare la presenza dell'outlier: 35 s - Tempo richiesto per l'analisi delle possibili implicazioni di questo outlier sui dati: 115 s)

A conferma di ciò, proprio quando Giulia aveva terminato il suo articolo e stava per chiudere la dashboard, si accorge che le è stata notificata la presenza di dati aggiornati. Infatti, la ASL di Bologna ha aggiornato il numero delle terapie intensive alle 17:25 perché tre pazienti appena dopo l'invio dei dati erano stati spostati, sfortunatamente, dalla terapia sub-intensiva a quella intensiva: la regione ha quindi prontamente aggiornato i dati e la dashboard ha di conseguenza avvisato i propri utenti, evitando che Giulia pubblicasse un articolo con dati errati. (Tempo richiesto per notare la notifica: 5 s - Tempo richiesto per l'analisi delle possibili implicazioni di questo nuovo aggiornamento sui dati: 95 s)

Task: 1, 2
Tempo totale: 7' 18''

\subsubsection*{Scenario 3: Analisi sull'andamento del tasso di letalità e dei numeri dell'epidemia nelle due settimane precedenti}
Francesco, ogni due settimane deve analizzare il tasso di letalità e come i numeri dell’epidemia si distribuiscono sulla popolazione per scrivere un articolo di giornale per un’importante editore italiano. Durante il periodo che precede la pubblicazione accede diverse volte alla dashboard per analizzare i dati di suo interesse. 
(Tempo stimato per l'apertura e il caricamento: 20 s)

Dopo aver inserito il periodo che vuole analizzare e di quali dati anagrafici vuole ottenere i valori (età, genere, impiego, malattie pregresse), la dashboard mostra a Francesco come questi sono distribuiti sul periodo sotto esame. 
I dati ottenuti gli permettono di scrivere un articolo preciso e approfondito riguardo all'andamento dell'epidemia Covid-19 in Italia. Inoltre, riesce a dare informazioni più dettagliate ai suoi lettori su chi sono le persone dietro quei numeri: la loro età, il loro genere, se hanno avuto malattie pregresse o meno oppure il loro impiego.
(Tempo richiesto per la selezione del periodo, dei dati anagrafici di interesse e il carimento: 12 s - Tempo richiesto per l'analisi dei dati 125 s)

Esplorando i dati ed ordinandoli per numero di decessi, ravvisa una possibile correlazione tra numero di decessi ed età: nell'intenzione di validare questa sua ipotesi, richiede alla dashboard che gli presenti con granularità maggiore e con aggregazioni più significative i valori relativi a questi due dati. 
Questa nuova visione della dashboard incuriosisce Francesco al punto da fargli emergere nuove domande tra cui qual è la relazione tra numero di decessi e impiego. Purtroppo il dato relativo all'impiego non è ancora disponibile. La dashboard avvisa Francesco che tale dato verrà inserito quanto prima non essendo ancora pervenuto per il periodo temporale richiesto.
(Tempo richiesto per la scoperta di alcune correlazioni, della comprensione di come impostare la dashboard per visualizzare i dati di suo interesse, l'impostazione sulla dashboard vera e propria e il caricamento: 210 s - Tempo richiesto per l'analisi dei dati 140 s)

Task svolti: 3, 4
Tempo totale: 8' 27''

\subsubsection*{Scenario 4: Analisi sulle differenze dell'andamento nelle regioni Emilia Romagna, Veneto e Molise nelle precedenti due settimane}
Matteo, il giornalista che nei suoi articoli confronta l'andamento della pandemia in regioni diverse, alle ore 17 si collega alla dashboard della Protezione Civile dalla sua postazione di lavoro a casa, in cui dispone di un comodo schermo curvo da 32". (Tempo stimato per l'apertura e il caricamento: 20 s)

In dettaglio, vuole scrivere l'articolo a cadenza bisettimanale in cui confronta i dati della pandemia nelle  regioni che si sono maggiormente distinte sullo scenario nazionale per l'evoluzione di una o più metriche: l'obiettivo di Matteo è quello di far emergere tratti comuni e aspetti distintivi che permettano di trarre conclusioni significative e di valore per comprendere al meglio il livello di gravità e rischio di ogni regione, anche alla luce delle situazioni in quelle non oggetto di esame.

Vuole recuperare le varie metriche pubblicate solitamente negli articoli giornalieri, ma sintetizzate nel periodo delle due settimane appena trascorse (il rapporto medio tra nuovi positivi e tamponi nel periodo, la media del numero di ospedalizzazioni giornaliere, decessi cumulati…).
Precisamente, per l'articolo corrente, intende concentrarsi sulle regioni Emilia Romagna, Veneto e Molise.
(Tempo di impostazione della dashboard per la visualizzazione delle tre regioni per il confronto: 15 s)

Procede quindi col richiedere alla dashboard di visualizzare contemporaneamente i dati della pandemia relativi alle tre regioni di interesse, in un formato chiaro e che gli permetta un agile confronto: la dashboard, immediatamente, gli restituisce calcoli automatici, come rapporti e differenze, tramite cui può comprendere, con dati alla mano, le reali differenze o somiglianze.
(Tempo richiesto per la selezione dei dati di interesse, delle metriche aggregate e il carimento: 20 s - Tempo richiesto per l'analisi dei dati 135 s) 

A Matteo pare di notare degli outlier (valori fuori linea rispetto a quelli precedenti) ma analizzando le note del giorno si accorge che in realtà i dati sono stati comunicati correttamente e che quindi non vi sono errori. Matteo può quindi procedere alla redazione del suo articolo. (Tempo richiesto per cercare le note e non trovarle: 30 s)

Task svolti: 5
Tempo totale: 3' 40''

\subsubsection*{Scenario 5: Analisi delle differenze dell'andamento dell'epidemia in Italia nel mese corrente rispetto ai due precedenti}
Dopo una settimana in cui Maurizio, il giornalista che si occupa degli articoli sull'andamento dell'epidemia Covid-19 in Italia, ha raccontato ai suoi lettori come la situazione si stava evolvendo giorno per giorno e dopo aver redatto anche l'articolo quotidiano, resta collegato sulla dashboard della Protezione Civile col suo portatile 15" per scrivere un ulteriore articolo, questa volta sul confronto fra il mese corrente e i precedenti due. (Tempo stimato per l'apertura e il caricamento: 0 s)

Maurizio inserisce i periodi temporali che vuole confrontare e la dashboard gli presenta i dati che sono stati raccolti, rendendoli facilmente confrontabili tra loro in modo tale che riesca a comprendere al meglio qual è la reale situazione dell’andamento dell’epidemia. In particolare, è interessato all'andamento del rapporto tra i positivi riscontrati sul numero di tamponi effettuati, del tasso di letalità, delle ospedalizzazioni e dell'occupazione delle terapie intensive.
(Tempo richiesto per la selezione dei periodi, la raccolta dei dati e l'impostazione della visualizzazione per il confronto: 35 s - Tempo richiesto per l'analisi dei dati 210 s)

Analizzando i dati dei mesi di settembre, ottobre e novembre nota che vi è una significativa diminuzione del numero di contagi con l'inasprimento delle misure restrittive soprattutto a fine novembre, esattamente due settimane dopo l'entrata in vigore del DPCM più recente. (Tempo richiesto per individuare la sezione con li provvedimenti normativi approvati in materia di contrasto all'epidemia, espansione e lettura dei punti salienti dei provvedimenti: 100 s - Tempo richiesto per l'analisi delle possibili correlazioni di questi provvedimenti sui dati: 140 s)

Task svolti: 6
Tempo totale: 8' 5''
