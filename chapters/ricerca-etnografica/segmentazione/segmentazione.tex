\subsection{Segmentazione}
Il 2020 passerà alla storia come uno dei periodi più complicati su numerosi fronti, da quello sanitario a quello sociale, 
da quello politico a quello economico: l'11 marzo 2020 l'Organizzazione Mondiale della Sanità (OMS) ha dichiarato che il 
focolaio internazionale di infezione da nuovo coronavirus SARS-CoV-2 può essere considerato una pandemia.\\
Tra i vari trend emersi, il giornalismo ha riacquisito in questo periodo storico una centralità fondamentale, rivelandosi 
essenziale per un'informazione rigorosa e consapevole dei cittadini.\\
Tuttavia, leggendo gli articoli pubblicati quotidianamente da varie testate nazionali e locali sull'andamento dei contagi 
e altri aspetti quantitativi legati alla pandemia Covid-19 in Italia, abbiamo notato come tutti questi avessero un comune 
denominatore: la sostanziale presentazione in maniera asettica dei dati comunicati dal Dipartimento della Protezione Civile 
(DPC) senza però approfondire come questi influiscano sulle vite dei cittadini italiani. Essendo questi articoli diventati 
ormai molto seguiti, c'è il rischio che non contestualizzando i dati presentati, come viene attualmente fatto, porti a 
situazioni di preoccupazione o rilassamento da parte dei cittadini ingiustificate e talvolta pericolose.\\
Conseguentemente, più forte è divenuta la necessità di strumenti tecnologici a supporto dell'attività di informazione 
realizzata dai giornalisti.\\
Il DPC, a partire dal 24 febbraio 2020, ha reso pubblico una dashboard (cruscotto informativo), dal nome ``COVID-19 Situazione Italia", 
al fine di comunicare i dati, aggiornati quotidianamente, sull'andamento della pandemia.\\
Analizzando questa dashboard, crediamo che non riesca a fornire un supporto significativo ai giornalisti che intendono pubblicare 
articoli sul tema: per questo motivo, il nostro impegno si è rivolto alla riprogettazione della dashboard, specificatamente per il 
segmento di utenza dei giornalisti, con l'obiettivo di rendere agevole e profonda la loro comprensione.\\
In dettaglio, i giornalisti cui ci riferiamo sono coloro che pubblicano articoli su andamento vari aspetti quantitativi della 
pandemia (contagi giornalieri, decessi, guariti, tamponi, occ. terapie intensive, ricoveri...), non necessariamente con 
background scientifico (liceo scientifico, o lauree in tema Scienze matematiche, fisiche e naturali, in sigla MM.FF.NN.).