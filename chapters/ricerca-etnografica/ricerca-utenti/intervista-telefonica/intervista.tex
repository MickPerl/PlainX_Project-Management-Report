Inizialmente avevamo a disposizione un solo contatto di una giornalista di \textit{SkyTG24}, ed è stata quindi contattata telefonicamente per un'intervista.
Di seguito vengono riportate le domande poste e, riassunte, le risposte ricevute\footnote{L'audio integrale dell'intervista e gli spezzoni con le risposte sono presenti tra i file consegnati per il progetto.}:
\begin{enumerate}
    \item \textbf{Domanda}: Quanto spesso usi la dashboard del DPC?\\\textbf{Risposta}: Quotidianamente, perché è necessaria al fine di tenere d'occhio l'andamento della pandemia. Viene costantemente aggiornata la pagina per verificare la presenza di nuovi dati.
    \item \textbf{Domanda}: Cosa ne pensi delle funzionalità e dell’impatto grafico della dashboard?\\\textbf{Risposta}: La dashboard è abbastanza intuitiva e funziona abbastanza bene, ma graficamente è poco accattivante, data la pervasione del colore nero.
    \item \textbf{Domanda}: Cosa ne pensi dei dati mostrati (parte 1)?\\\textbf{Risposta}: Sono presenti solo dati assoluti e sarebbe invece interessante offrire dati rapportati (per esempio: $tasso\; di\; positivit\grave{a} = \frac{tamponi\; positivi}{tamponi\; effettuati}$, in un giorno)
    \item \textbf{Domanda}: Come analizzate voi in SkyTG24 i dati della pandemia (parte 1)?\\\textbf{Risposta}: Si è consolidato un processo di business che fornisce una lista di valori di metriche automaticamente calcolate (come avverrebbe in un foglio di calcolo) a partire da dati ricevuti da organizzazioni esterne (per esempio dal DPC, dall'Istituto Superiore di Sanità (ISS)).
    \item \textbf{Domanda}: Cosa ne pensi dei dati mostrati (parte 2)?\\\textbf{Risposta}: I dati mostrati, con l'avanzare delle settimane e dei mesi, diventano sempre più grandi e hanno sempre più meno senso, se presi in assoluto. Dati come il numero degli attuali positivi o dei guariti, purtroppo, non aiutano molto nella comprensione dell'andamento. Dati come gli attuali ricoveri in terapia intensiva e ospedalizzazioni non ci sono (in verità ci sarebbero sia quelli a livello nazionale, ma non vengono mostrati, sia quelli a livello regionale, ma sono nascosti). Sarebbe utile, inoltre, avere una correlazione dei dati del DPC con quelli dell'ISS, che raccoglie informazioni quali l'età mediana e la distribuzioni rispetto alla popolazione.
    \item \textbf{Domanda}: Cosa ne pensi dei grafici e della mappa mostrati?\\\textbf{Risposta}: Sono grafici poco spiegati, che presentano delle curve di andamento ma non si capisce come interpretarle. Inoltre, la mappa dell'Italia, che occupa la maggior parte della dashboard non ha un funzionamento chiaro.
    \item \textbf{Domanda}: Come analizzate voi in SkyTG24 i dati della pandemia (parte 2)?\\\textbf{Risposta}: Vengono utilizzati i bollettini quotidiani del DPC e informazioni provenienti da un sistema realizzato da dei \textit{Data journalist} che automaticamente invia una mail alla loro redazione.
    \item \textbf{Domanda}: Sarebbe comoda una dashboard unica?\\\textbf{Risposta}: Ovviamente sì. L'importante è che i dati siano comprensibili, che siano spiegati; i giornalisti ormai sanno abbastanza destreggiarsi fra le varie terminologie e interpretazioni dei dati ma ai cittadini questo manca. Tutti dovrebbero poter comprendere i dati presentati dalla dashboard.
    \item \textbf{Domanda}: La difficoltà nella lettura dei dati inficia la qualità degli articoli prodotti?\\\textbf{Risposta}: Sì e anche le istituzioni dovrebbero fornire gli strumenti per analizzarli con più facilità.
    \item La giornalista ci informa a riguardo di un'altra fonte di informazione.\\\textbf{Risposta}: Esiste un'altra fonte, che presenta dati e quindi dei cruscotti sull'andamento della pandemia: è l'Agenzia Nazionale per i Servizi Sanitari Regionali (AGENAS). Fornisce informazioni sullo stato delle terapie intensive, ospedali e pronti soccorsi nelle varie regioni e a livello nazionale. Perché questa fonte preziosa non è pubblicizzata a dovere? Perché questi dati non sono integrati nella dashboard del DPC?
    \item \textbf{Domanda}: Sarebbe interessante la possibilità di fornire un confronto tra nazioni, regioni, province e i loro indicatori?\\\textbf{Risposta}: Ha poco senso fornire il confronto fra nazioni se la dashboard è sull'Italia. È invece utile fornire il confronto tra regioni (tra province meno, in quanto le province non hanno colori - rosso, arancione, giallo - a differenza delle regioni), seppur potrebbe esserci mancanza di dati in caso le istituzioni non li forniscano.
    \item \textbf{Domanda}: Sarebbe interessante un confronto rispetto al numero di tamponi effettuati?\\\textbf{Risposta}: Potrebbe essere utile ma c'è da porre attenzione su un confronto di questo tipo: le regioni potrebbero contare fra i tamponi effettuati anche i tamponi rapidi, non solo quelli molecolari. Un confronto di questo tipo potrebbe essere fuorviante.
    \item \textbf{Domanda}: Sarebbe interessante un confronto rispetto al numero dei decessi?\\\textbf{Risposta}: Sì, ma correlati di altri dati su età, sesso, malattie pregresse e dove abitano.
\end{enumerate}