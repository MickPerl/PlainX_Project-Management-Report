\section{Inspection}
\label{s:inspection}

In questa sezione vengono mostrate le modifiche apportate al wireframe di cui disponevamo, all'inizio della valutazione di tipo \textit{inspection}, per poi condurlo a quanto presentato nella \hyperref[s:wireframe]{Sezione 4.6 Wireframe}.\\
In allegato a questo documento vi è una cartella \texttt{wireframe} in cui vi sono tutti i wireframe raccolti e la loro evoluzione ottenuta dopo la valutazione di tipo \textit{inspection} e anche dopo quella di tipo \textit{testing}.

\subsection{Cognitive walkthrough}
\label{ss:cognitive-walkthrough}

Nella seguente sotto-sezione viene discusso come abbiamo effettuato la valutazione della nostra riprogettazione su dei wireframe che avevamo a quel momento a disposizione.\\
I \textit{Cognitive Walkthrough} sono stati eseguiti in ordine di task, dal primo all'ultimo. Alla fine di ciascuno, se è stato ritenuto necessario apportare delle modifiche lo si è indicato specificando quanto svolto per poter permettere il corretto svolgimento dei task.
Al termine di ogni correzione è stato salvato lo stato dei wireframe e usato questi con le modifiche apportate come baseline per i successivi \textit{Cognitive Walkthrough}.\\

\subfile{cognitive-walkthrough/task1}
\subfile{cognitive-walkthrough/task2}
\subfile{cognitive-walkthrough/task3}
\subfile{cognitive-walkthrough/task4}
\subfile{cognitive-walkthrough/task5}
\subfile{cognitive-walkthrough/task6}

\subsection{Informal Action Analysis}
\label{ss:informal-action-analysis}
Abbiamo deciso di affrontare l'\textit{action analysis} come modalità di valutazione per comprendere se i task sono portabili a termine con successo da parte del segmento di utenza della dashboard in tempi da noi ritenuti accettabili.\\
L'\textit{action analysis} da noi scelta è, come indicato nel titolo, di tipo \textit{informal}, per permetterci a grandi linee di comprendere le tempistiche di esecuzione dei task.\\
Abbiamo ritenuto opportuno non assegnare dei tempi più o meno precisi per lo svolgimento di ogni azione atomica che compone un task ma bensì dei pesi, come indicato di seguito:
\begin{itemize}
    \item Molto breve (circa 0-2 s)
    \item Breve (circa 3-5 s) 
    \item Medio (circa 6-10 s) 
    \item Lungo (circa 11-20 s) 
    \item Molto lungo ($>$ 20 s)
\end{itemize}

\subfile{informal-action-analysis/task1}
\subfile{informal-action-analysis/task2}
\subfile{informal-action-analysis/task3}
\subfile{informal-action-analysis/task4}
\subfile{informal-action-analysis/task5}
\subfile{informal-action-analysis/task6}

\subsection{Heuristic Analysis}
\label{ss:heuristic-analysis}
Dopo aver valutato in \hyperref[s:revisione-usabilita-esperti]{Sezione 2.1 Revisione dell'usabilità da parte di esperti} la dashboard oggetto di riprogettazione attraverso delle linee guida da noi selezionate, anche in questo caso le abbiamo sfruttate per verificare se la nostra versione le rispettava.
Ogni qual volta la nostra dashboard non aderiva alle linee guida, veniva aggiornata l'interfaccia e poi salvata una nuova versione del wireframe, come descritto nell'introduzione di \hyperref[s:inspection]{questa sezione}.
\subfile{heuristic-analysis/heuristic-analysis}