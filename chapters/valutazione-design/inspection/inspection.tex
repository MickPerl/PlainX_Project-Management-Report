\section{Inspection}
\label{s:inspection}

In questa sezione vengono mostrate le modifiche apportate al wireframe di cui disponevamo, all'inizio della valutazione di tipo \textit{inspection}, per poi condurlo a quanto presentato nella \hyperref[s:wireframe]{Sezione 4.6 Wireframe}.\\
In allegato a questo documento vi è una cartella \texttt{wireframe} in cui vi sono tutti i wireframe raccolti e la loro evoluzione ottenuta dopo la valutazione di tipo \textit{inspection} e anche dopo quella di tipo \textit{testing}.

\subsection{Cognitive walkthrough}
\label{ss:cognitive-walkthrough}

Nella seguente sotto-sezione viene discusso come abbiamo effettuato la valutazione della nostra riprogettazione su dei wireframe che avevamo a quel momento a disposizione.\\
I \textit{Cognitive Walkthrough} sono stati eseguiti in ordine di task, dal primo all'ultimo. Alla fine di ciascuno, se è stato ritenuto necessario apportare delle modifiche lo si è indicato specificando quanto svolto per poter permettere il corretto svolgimento dei task.
Al termine di ogni correzione è stato salvato lo stato dei wireframe e usato questi con le modifiche apportate come baseline per i successivi \textit{Cognitive Walkthrough}.\\

\subfile{cognitive-walkthrough/task1}
\subfile{cognitive-walkthrough/task2}
\subfile{cognitive-walkthrough/task3}
\subfile{cognitive-walkthrough/task4}
\subfile{cognitive-walkthrough/task5}
\subfile{cognitive-walkthrough/task6}

\subsection{Informal Action Analysis}
\label{ss:informal-action-analysis}
Peso delle azioni:
\begin{itemize}
    \item Molto breve (0-2 s) (1-2 azioni atomiche)
    \item Breve (3-5s) (3-4 azioni atomiche)
    \item Medio (6-10s) (5-7 azioni atomiche)
    \item Lungo (11-20s) (8-10 azioni atomiche)
    \item Molto lungo (> 20 s) (> 10 azioni atomiche)
\end{itemize}

\subfile{informal-action-analysis/task1}
\subfile{informal-action-analysis/task2}
\subfile{informal-action-analysis/task3}
\subfile{informal-action-analysis/task4}
\subfile{informal-action-analysis/task5}
\subfile{informal-action-analysis/task6}

\subsection{Heuristic Analysis}
\label{ss:heuristic-analysis}

\subfile{heuristic-analysis/heuristic-analysis}