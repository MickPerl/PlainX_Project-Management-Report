\section{Inspection}
\label{s:inspection}

In questa sezione vengono mostrate le modifiche apportate al wireframe di cui disponevamo, all'inizio della valutazione di tipo \textit{inspection}, per poi condurlo a quanto presentato nella \hyperref[s:wireframe]{Sezione 4.6 Wireframe}.\\
In allegato a questo documento vi è una cartella \texttt{wireframe} in cui vi sono tutti i wireframe raccolti e la loro evoluzione a seguito della valutazione di tipo \textit{inspection} e di quella di tipo \textit{testing}.

\subsection{Cognitive walkthrough}
\label{ss:cognitive-walkthrough}

Nella seguente sotto-sezione viene presentata la valutazione che abbiamo realizzato sui wireframe di cui disponevamo prima di iniziare la valutazione di tipo \textit{inspection}.\\
I \textit{Cognitive Walkthrough} (CW) sono stati eseguiti in ordine di task, dal primo all'ultimo: alla fine di qualcuno, abbiamo ritenuto necessario apportare delle modifiche all'interfaccia, documentandole sempre opportunatamente; inoltre, i CW successivi sono stati eseguiti sempre sull'ultima versione dell'interfaccia migliorata.

\subfile{cognitive-walkthrough/task1}
\subfile{cognitive-walkthrough/task2}
\subfile{cognitive-walkthrough/task3}
\subfile{cognitive-walkthrough/task4}
\subfile{cognitive-walkthrough/task5}
\subfile{cognitive-walkthrough/task6}

\subsection{Informal Action Analysis}
\label{ss:informal-action-analysis}
Abbiamo deciso di affrontare l'\textit{action analysis} come modalità di valutazione per comprendere se i task sono portabili a termine con successo da parte del segmento di utenza della dashboard in tempi da noi ritenuti accettabili.\\
L'\textit{action analysis} da noi scelta è, come indicato nel titolo, di tipo \textit{informal}, per permetterci a grandi linee di comprendere le tempistiche di esecuzione dei task.\\
Abbiamo ritenuto opportuno assegnare dei tempi più o meno precisi per lo svolgimento di ogni azione atomica che compone un task ma bensì dei pesi, come indicato di seguito:
\begin{itemize}
    \item molto breve (circa 0-2 s);
    \item breve (circa 3-5 s);
    \item medio (circa 6-10 s);
    \item lungo (circa 11-20 s);
    \item molto lungo ($>$ 20 s).
\end{itemize}

\subfile{informal-action-analysis/task1}
\subfile{informal-action-analysis/task2}
\subfile{informal-action-analysis/task3}
\subfile{informal-action-analysis/task4}
\subfile{informal-action-analysis/task5}
\subfile{informal-action-analysis/task6}

\subsection{Heuristic Analysis}
\label{ss:heuristic-analysis}
Abbiamo valutato sull'interfaccia cui siamo giunti le medesime linee guida su cui abbiamo valutato l'interfaccia originale nella \hyperref[s:revisione-usabilita-esperti]{Sezione 2.1 Revisione dell'usabilità da parte di esperti}: a fronte di ogni violazione, abbiamo immediatamente migliorato l'interfaccia e solo in seguito proceduto con la valutazione delle linee guida seguenti.
\subfile{heuristic-analysis/heuristic-analysis}