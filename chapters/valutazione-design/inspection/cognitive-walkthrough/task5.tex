
\subsubsection{Task 5}
\label{sss:cw-task-5}

Il giornalista di testata nazionale accede alla dashboard del DPC attraverso il browser.\\
Vuole confrontare l'andamento della pandemia in diverse regioni e per ciascuna di esse raccogliere le seguenti metriche: nuovi positivi, tasso di positività, tamponi effettuati, decessi, i dimessi e i guariti, gli attuali positivi, nuovi ingressi in terapia intensiva, tasso di occupazione delle terapie intensive, capacità delle terapie intensive, le ospedalizzazioni, numero di deceduti e il totale dei casi.\\
Pertanto, si reca nella sezione ``Confronto fra regioni" e procede col selezionare sulla mappa le regioni che intende confrontare.
Individua alcune delle metriche di interesse nei grafici a linea della schermata e opera sul pulsante con l'icona a forma di ghiera presente nella barra di navigazione per visualizzare alcune di suo interesse ma non ancora presenti nella schermata.
Sposta il cursore sulle curve di cui vuole avere maggiori dettagli e legge il tooltip che appare, riuscendo così a ricavare conclusioni sul confronto.