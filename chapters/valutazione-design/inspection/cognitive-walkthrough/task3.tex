\subsubsection{Task 3}
\label{sss:cw-task-3}

Il giornalista di testata nazionale accede alla dashboard del DPC attraverso il browser.
Vuole raccogliere le seguenti metriche: numero di deceduti e il totale dei casi al giorno di inizio e a quello di fine del periodo.
Essendo interessato a valutare l'andamento di una metrica relativamente ad un periodo di tempo, ecco che si dirige sulla schermata ``Analisi di periodi" (precedentemente ``Confronto tra periodi"), il cui titolo sembra essere particolarmente appropriato per il suo bisogno.
Utilizza il widget calendario attivabile mediante click sull'icona a forma di calendario o sulla data stessa per indicare la data di inizio e fine periodo, infine osserva sul grafico ``Decessi" i dati a cui è interessato.

\begin{bclogo}{Aggiustamenti apportati dopo il Cognitive Walkthrough sul Task 3}
    Questo task può essere completato con successo dal giornalista sulla dashboard riprogettata, tuttavia la sua rilettura ci ha dato l'idea di includere anche la metrica "Tasso di letalità", ossia l'aggregato ottenibile rapportando il numero dei decessi al numero dei casi totali: così facendo, l'utente è sollevato dalla necessità di compiere i calcoli e ottiene il valore della metrica di interesse con un semplice click.
    In dettaglio, per questioni di spazio, abbiamo aggiunto un ulteriore pulsante ``Seleziona metriche" nella barra di navigazione tramite cui l'utente può scegliere quali componenti grafiche devono apparire nella schermata.
\end{bclogo}