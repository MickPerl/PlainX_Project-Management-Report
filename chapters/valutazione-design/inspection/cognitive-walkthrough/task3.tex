\subsubsection{Task 3}
\label{sss:cw-task-3}

Il giornalista di testata nazionale accede alla dashboard del DPC attraverso il browser.\\
Vuole raccogliere, circa il giorno di inizio e quello di fine di un certo periodo, il numero di deceduti e i casi totali.
Essendo interessato a valutare l'andamento di una metrica con riferimento ad un periodo di tempo, ecco che si dirige sulla schermata ``Confronto tra periodi", il cui titolo sembra essere particolarmente appropriato per il suo bisogno.\\
Utilizza il widget calendario attivabile mediante click sull'icona del calendario o sulla data stessa per indicare la finestra temporale di suo interesse, quindi osserva la curva del grafico ``Decessi".

\begin{bclogo}{Nuova iterazione a seguito del CW sul Task 3}
    Questo task può essere completato con successo dal giornalista sulla dashboard riprogettata, tuttavia, abbiamo ravvisato la possibilità di un ulteriore miglioramento, consistente nell'inclusione della metrica ``Tasso di letalità", ossia l'aggregato ottenibile rapportando il numero dei decessi al numero dei casi totali: così facendo, l'utente è sollevato dalla necessità di compiere i calcoli e ottiene il valore della metrica di interesse con un semplice click.
    In dettaglio, per gestire efficacemente lo spazio, abbiamo aggiunto un ulteriore pulsante con un'icona a forma di ghiera nella barra di navigazione tramite cui l'utente può scegliere quali componenti grafiche devono apparire nella schermata.
\end{bclogo}