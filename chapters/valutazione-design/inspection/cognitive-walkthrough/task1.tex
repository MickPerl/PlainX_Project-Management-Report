\subsubsection{Task 1}
\label{sss:cw-task-1}
Il giornalista di testata nazionale/locale accede alla dashboard del DPC attraverso il browser.
Vuole raccogliere, relativamente alla giornata odierna, le seguenti metriche sull'andamento della pandemia a livello nazionale: i nuovi positivi, il tasso di positività, il numero di tamponi effettuati, i decessi, i dimessi e i guariti, gli attuali positivi.
Vuole inoltre sapere quali sono i colori delle regioni, ai sensi dei \href{https://www.gazzettaufficiale.it/eli/gu/2020/11/04/275/so/41/sg/pdf}{DPCM 03/11/2020} e \href{https://www.gazzettaufficiale.it/eli/id/2021/01/15/21A00221/sg}{DPCM 14/01/2021}.\\
Raccoglie le metriche indicate nei modi seguenti:
\begin{enumerate}[label=\alph*.]
    \item Nuovi positivi: il giornalista, per esperienza con l'attuale e con altre dashboard, sa che il dato dei nuovi positivi viene indicato solitamente come variazione del totale dei casi dal giorno precedente. Tuttavia, essendo questa la sua prima visita alla nuova dashboard, legge le etichette di tutti gli altri box per essere sicuro che non sia indicato in altre parti. Non trovando un esplicito riferimento altrove, assume che anche in questa dashboard sia mantenuta la convenzione di cui sopra;
    \item Tasso di positività e numero di tamponi effettuati: il giornalista individua a colpo d'occhio l'etichetta ``Tasso di positività" nel box ``Totale casi" e legge il valore subito sottostante. Nonostante ciò, si accorge della presenza di un pulsante al di sotto del box con la stessa etichetta che, alla sua pressione, permette la visualizzazione di un grafico con l'andamento temporale del tasso di positività dall'inizio della pandemia. Per curiosità, posizionando il cursore al di sopra della curva rappresentata nel grafico, scopre che viene visualizzato un tooltip con il tasso effettivo alla posizione del cursore e i tamponi effettuati a quella data;
    \item Decessi: il giornalista individua il box ``Decessi" a colpo d'occhio e, notando due numeri con magnitudo completamente diversi, capisce che il minore dei due è quello dei decessi odierni;
    \item Dimessi e guariti: il giornalista individua il box ``Dimessi e guariti" a colpo d'occhio e, notando due numeri con magnitudo completamente diversi, capisce che il minore dei due è quello dei dimessi e guariti odierni;
    \item Attuali positivi: il giornalista individua il box dei ``Dimessi e guariti" a colpo d'occhio e vedendo due numeri con magnitudo completamente diversi, comprende che il minore dei due è quello dei dimessi e guariti odierni;
    \item Colori delle regioni: il giornalista nota che la mappa presente nella schermata è divisa in regioni e ogni regione ha un colore fra arancio, giallo, rosso. Di conseguenza assume che quelli siano i colori relativi ai DPCM sopracitati. Nota inoltre che, nella mappa dell'Italia a lato della schermata, vi è un icona a forma di ingranaggio che ricorda l'idea di ``personalizzazione" o ``impostazioni" come nelle principali applicazioni e sistemi operativi. Cliccandoci sopra, vede che è selezionata la voce ``Monitoraggio regioni", avendo la conferma che la colorazione delle regioni è effettivamente quella introdotti dai DPCM. \label{cw1:f}
\end{enumerate}
Una volta raccolti i valori delle metriche, verifica come questi si rapportano fra loro e infine li vuole confrontare con i dati della giornata precedente: ha compreso come raccogliere le metriche ma ora deve vedere i valori di quelle antecedenti.
Per fare ciò, nota che nella barra di navigazione in alto vi è la data odierna e, subito accanto, un'icona di calendario, una freccia verso sinistra che sembra indicare la possibilità di cambiare la visualizzazione con i dati della giornata precedente.
Premendo tale pulsante nota che la data cambia e di conseguenza anche tutti i dati visualizzati.
Ha quindi conferma del cambio dei dati e può ora procedere alla redazione del suo articolo.

\begin{bclogo}{Aggiustamenti apportati dopo il Cognitive Walkthrough sul Task 1}
    La descrizione appena presentata è stata ritenuta carente in tutti i punti fuori che nell'ultimo, \ref{cw1:f}.
    Infatti, nei primi punti, l'incertezza del giornalista e il tempo impiegato per leggere tutti i box ci ha spinto ad effettuare una nuova iterazione, concretizzatasi nell'aggiunta di un'etichetta che esplicita ciò a cui si riferiscono le variazioni di ogni box.
    Per questo motivo, sopra ad ogni valore odierno dei box nella schermata ``Panoramica" (precedentemente ``Situazione odierna nazionale"), viene aggiunta una nuova etichetta come già fatto per gli altri valori nei box medesimi.
\end{bclogo}