\subsubsection{Task 1}
\label{sss:cw-task-1}
Il giornalista di testata nazionale/locale accede alla dashboard del DPC attraverso il browser.
Vuole raccogliere le seguenti metriche sull'andamento della pandemia a livello nazionale per la giornata odierna: nuovi positivi, tasso di positività, tamponi effettuati, decessi, i dimessi e i guariti, gli attuali positivi.
Vuole inoltre sapere quali sono i colori (secondo i DPCM 03/11/2020 e DPCM 16/01/2021) delle regioni.
Raccoglie le metriche indicate nei modi seguenti:
\begin{enumerate}[label=\alph*.]
    \item Nuovi positivi: il giornalista, per esperienza su altre dashboard, sa che il dato dei nuovi positivi viene scritto solitamente come variazione nel box dei Totale casi; tuttavia, essendo la prima visita a questa dashboard, legge le etichette di tutti gli altri box, per essere sicuro che non sia indicato in altre parti; non trovando un esplicito riferimento altrove, assume che anche in questa dashboard sia mantenuta la convenzione di cui sopra;
    \item Tasso di positività e tamponi effettuati: il giornalista, individua a colpo d'occhio l'etichetta "Tasso di positività" nel box "Totale casi" e legge il valore subito sottostante, tuttavia si accorge della presenza di un bottone sotto il box con la stessa etichetta; premendolo viene visualizzato un grafico con l'andamento temporale del tasso di positività dall'inizio della pandemia e per curiosità, posizionando il cursore al di sopra della curva appare un tooltip che visualizza il tasso effettivo alla posizione del cursore e i tamponi effettuati a quella data;
    \item Decessi: il giornalista individua il box dei "Decessi" a colpo d'occhio e vedendo due numeri con magnitudo completamente diversi, capisce che il minore dei due è quello dei decessi odierni;
    \item Dimessi e guariti: il giornalista individua il box dei "Dimessi e guariti" a colpo d'occhio e vedendo due numeri con magnitudo completamente diversi, capisce che il minore dei due è quello dei dimessi e guariti odierni;
    \item Attuali positivi: il giornalista individua il box dei "Dimessi e guariti" a colpo d'occhio e vedendo due numeri con magnitudo completamente diversi, capisce che il minore dei due è quello dei dimessi e guariti odierni;
    \item Colori delle regioni: il giornalista nota che la mappa presente nella schermata è divisa in regioni e ogni regione ha un colore fra arancio, giallo, rosso. Di conseguenza assume che quelli siano i colori relativi ai DPCM 03/11/2020 e DPCM 16/01/2021. Nota che nella mappa vi è un icona a forma di ingranaggio, che ricorda l'idea di "personalizzazione" o "impostazioni" come nelle principali applicazioni e sistemi operativi e, cliccandoci sopra, vede che è selezionata la voce "Monitoraggio regioni", avendo la conferma che i colori visualizzati sono effettivamente quelli introdotti dai DPCM.
\end{enumerate}
Una volta raccolti i valori delle metriche verifica come questi si rapportano fra di loro e infine li confronta con i dati della giornata precedente: ha compreso come raccogliere le metriche ma deve ora vedere quelle della giornata precedente.
Per fare ciò, nota che nella barra di navigazione in alto vi è la data odierna e subito accanto un'icona di calendario e una freccia verso sinistra che sembra indicare la possibilità di cambiare la visualizzazione con i dati della giornata precedente.
Premendo tale pulsante nota che la data cambia e di conseguenza anche tutti i dati visualizzati.
Ha conferma quindi del cambio dei dati e può ora procedere alla redazione del suo articolo.

\begin{bclogo}{Aggiustamenti apportati dopo il Cognitive Walkthrough sul Task 1}
    Il CW risulta carente in tutti i punti fuori che l'ultimo, f).
    Infatti, nei primi punti, l'incertezza del giornalista e il tempo che ha impiegato per leggere tutti i box ci ha spinto ad effettuare una nuova ITERAZIONE che si è concretizzata nell'aggiunta di un'etichetta che esplicita ciò a cui si riferiscono le variazioni di ogni box.
    Per questo motivo, sopra ad ogni valore odierno dei box nella schermata "Situazione  odierna nazionale" viene aggiunta una nuova etichetta come già fatto per gli altri valori nei box medesimi.
\end{bclogo}