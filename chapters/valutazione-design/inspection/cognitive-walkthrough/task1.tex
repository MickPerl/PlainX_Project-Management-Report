\subsubsection{Task 1}
\label{sss:cw-task-1}
Il giornalista di testata nazionale/locale accede alla dashboard del DPC attraverso il browser.\\
Vuole raccogliere, relativamente alla giornata odierna, le seguenti metriche sull'andamento della pandemia a livello nazionale: i nuovi positivi, il tasso di positività, il numero di tamponi effettuati, i decessi, i dimessi e i guariti, gli attuali positivi.\\
Vuole inoltre sapere quali sono i colori delle regioni, ai sensi dei \href{https://www.gazzettaufficiale.it/eli/gu/2020/11/04/275/so/41/sg/pdf}{DPCM 03/11/2020} e \href{https://www.gazzettaufficiale.it/eli/id/2021/01/15/21A00221/sg}{DPCM 14/01/2021}.\\
Raccoglie le metriche indicate nei modi seguenti:
\begin{enumerate}
    \item nuovi positivi: il giornalista, sulla base dell'esperienza che ha con l'interfaccia attuale della dashboard del DPC nonché di altre dashboard, sa che il dato dei nuovi positivi viene indicato solitamente come variazione del totale dei casi dal giorno precedente. Tuttavia, essendo questa la sua prima visita alla nuova dashboard, legge le etichette di tutti gli altri box per essere sicuro che non sia indicato in altre parti: non trovando un esplicito riferimento altrove, assume che anche in questa dashboard sia mantenuta la convenzione di cui sopra;
    \item tasso di positività e numero di tamponi effettuati: il giornalista individua a colpo d'occhio l'etichetta ``Tasso di positività" nel box ``Totale casi" e legge il valore subito sottostante; inoltre, si accorge della presenza di un pulsante con la scritta ``Tasso di positività" alla base del box in oggetto, al cui click, permette la visualizzazione di un grafico dell'andamento temporale del tasso di positività dall'inizio della pandemia. Per curiosità, posizionando il cursore al di sopra della curva rappresentata nel grafico, scopre che viene visualizzato un tooltip con il tasso di positività e il numero di tamponi del giorno su cui è posto il cursore;
    \item decessi: il giornalista individua il box ``Decessi" a colpo d'occhio e, notando due numeri assoluti con magnitudo completamente diverse, capisce che il minore dei due è quello dei decessi odierni;
    \item dimessi e guariti: il giornalista individua il box ``Dimessi e guariti" a colpo d'occhio e, notando due numeri assoluti con magnitudo completamente diversi, capisce che il minore dei due è quello dei dimessi e guariti odierni;
    \item attuali positivi: il giornalista individua il box dei ``Attuali positivi" a colpo d'occhio e vedendo due numeri assoluti con magnitudo completamente diversi, comprende che il maggiore dei due è quello degli attuali positivi;
    \item colori delle regioni: il giornalista nota che la mappa presente nella schermata è divisa in regioni e ogni regione ha un colore diverso; nota anche, sulla destra, una legenda che indica i livelli della metrica in base a cui le regioni sono colorate: trattasi di ``Monitoraggio regioni'', per cui ha la conferma che trattasi della divisione in zone ai sensi dei DPCM sopra citati. \label{cw1:f}
\end{enumerate}
Una volta raccolti i valori delle metriche, ne valuta i rapporti e decide di confrontarli con quelli della giornata precedente. Per fare ciò, nota nell'intestazione della schermata una scritta con la data odierna, al cui fianco è posto un pulsante con l'icona del calendario e due frecce: comprende che, interagendo con questi pulsanti, può selezionare una data diverso o andare alla successiva/precedente rispetto a quella corrente; la sua intuizione viene confermata quando clicca sulla freccia a sinistra e la data presentata viene aggiornata al giorno precedente e, conseguentemente, anche i dati visualizzati.

\begin{bclogo}{Nuova iterazione a seguito del CW sul Task 1}
    Abbiamo ravvisato margini di miglioramento dell'interfaccia per l'esecuzione di tutte le azioni, fuorché l'\hyperref[cw1:f]{ultima}: in dettaglio, l'incertezza del giornalista e il tempo impiegato per leggere tutti i box ci ha spinto ad effettuare una nuova iterazione, concretizzatasi nell'aggiunta, in ogni box della schermata ``Situazione odierna nazionale" (poi rinominata in ``Panoramica"), di un'etichetta soprastante le variazioni che ne espliciti il portato informativo.
\end{bclogo}