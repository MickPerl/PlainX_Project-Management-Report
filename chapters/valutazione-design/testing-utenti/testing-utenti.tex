\section{Testing degli utenti}
\label{s:testing-utenti}

\subsection{Progettazione del test}
\label{ss:vd-progettazione-test}
Per testare la nostra riprogettazione, abbiamo ritenuto opportuno perseguire con l'approccio Discount usability testing già utilizzato in \hyperref[s:verifica-risorse-esistenti-testing-utenti]{Sezione 2.3}.\\
In questo caso, abbiamo utilizzato una strategia di test di tipo formativa, in quanto il nostro obiettivo è quello di trovare quanti più problemi di usabilità possibili grazie agli utenti dei test.\\
L'ambito del test è stato orizzontale, dal momento che quello a nostra disposizione è solamente un wireframe con interazioni sufficienti al solo scopo esemplificativo: in altre parole, le interazioni presenti sono volte a spiegare il funzionamento delle varie componenti grafiche, mentre non sempre permettono di portare a termine i task.\\
Il setting in cui verranno condotti i test ha visto l'utilizzo di software di videoconferenza. Gli assistenti del test siamo stati noi stessi, in qualità di membri del team di design. La selezione dei partecipanti avviene mediante l'invio di una mail agli indirizzi dei giornalisti precedentemente contattati in occasione della ricerca etnografica e del testing sulle risorse esistenti.\\
I dati che sono stati raccolti hanno natura qualitativa: in dettaglio, abbiamo chiesto ad ogni giornalista di simulare l'esecuzione di certi task mediante il wireframe e di esprimerci le sue impressioni a riguardo. L'output del test si è concretizzato, pertanto, in un insieme di dichiarazioni circa gli aspetti migliori dell'interfaccia progettata, nonché le criticità da risolvere.\\
Dato l'esiguo numero di giornalisti che si sono offerti per il test (3) e la natura qualitativa dei dati raccolti, abbiamo deciso di analizzarli applicando il buon senso.\\
Per i task stati sottoposti a ciascun tester, abbiamo attinto da quelli già selezionati nella prima fase di testing sulle risorse esistenti:
\begin{enumerate}
    \item calcolare il tasso di positività (calcolato come il rapporto tra nuovi positivi e i tamponi effettuati) relativo al 17 novembre 2020;
    \item valutare l'occupazione dei reparti di terapia intensiva al 17 novembre 2020;
    \item calcolare il tasso di letalità (calcolato come il rapporto tra numero di deceduti e numero di casi) nel mese di aprile 2020;
    \item calcolare il numero totale dei casi e il numero dei decessi per diverse fasce di età;
    \item confrontare l'andamento dei nuovi positivi giornalieri tra Emilia Romagna e Lombardia;
    \item confrontare la media dei nuovi positivi giornalieri tra il mese di Aprile e il mese di Maggio.
\end{enumerate}

\subsection{Selezione e preparazione degli assistenti}
\label{ss:vd-selezione-preparazione-assistenti}
Come riportato sopra, il test non ha previsto assistenti esterni, bensì è stato svolto interamente dai noi stessi, in qualità di membri del team di design. Abbiamo, a tal fine, aggiunto interazioni al Wireframe, cosicché sia maggiormente esplicativo e realistico per i tester, nonché applicato le nozioni apprese durante il corso di Usability \& User Experience Design.

\subsection{Test pilota}
\label{ss:test-pilota}
Il test pilota è stato condotto da noi stessi, in qualità di membri del team, tramite cui si è validato il protocollo definito in \hyperref[ss:selezione-preparazione-assistenti]{Sezione 5.2}: in dettaglio, si è verificata l'esistenza del c.d. happy path, il corretto funzionamento della piattaforma di videoconferenza scelta, il timing totale del test che ammonta a venti minuti e l'appropriatezza delle richieste.\\
L'happy path riscontrato per ogni task è riportato di seguito:
\begin{enumerate}
    \item calcolare il tasso di positività (calcolato come il rapporto tra nuovi positivi e i tamponi effettuati) relativo al 17 novembre 2020;
    \begin{enumerate}
        \item apertura dashboard;
        \item selezione data del 17 novembre 2020 dal widget calendario;
        \item lettura del numero sottostante l'etichetta "Tasso di positività" presente nel 1° box ("Totale casi");
    \end{enumerate}
    \item valutare l'occupazione delle strutture sanitare (tasso di occupazione dei reparti di terapia intensiva) al 17 novembre '20;
    \begin{enumerate}
        \item apertura dashboard;
        \item selezione data del 17 novembre 2020 dal widget calendario;
        \item click sul bottone "Tasso occupazione" presente alla base del 2° box ("Terapie intensive");
        \item lettura della percentuale;
    \end{enumerate}
    \item calcolare il tasso di letalità (calcolato come il rapporto tra il numero dei deceduti e il numero dei casi) medio nel mese di aprile;
    \begin{enumerate}
        \item apertura dashboard;
        \item click su link alla sezione "Analisi di periodi";
        \item selezione del mese di aprile nel primo widget calendario;
        \item selezione del bottone "Seleziona metriche" e trascinamento etichetta "Tasso di letalità" all'interno della schermata;
        \item lettura del valore numerico;
    \end{enumerate}
    \item analizzare la distribuzione del totale dei casi e dei decessi per diverse fasce di età
    \begin{enumerate}
        \item apertura dashboard;
        \item click su link alla sezione "Distribuzione su dati anagrafici";
        \item lettura del grafico "Distribuzione dei casi totali per fasce d’età";
        \item selezione del bottone "Seleziona metriche" e trascinamento etichetta "Distribuzioni dei decessi per fasce d'età" all'interno della schermata;
        \item lettura del grafico "Distribuzione dei decessi per fasce d’età";
    \end{enumerate}
    \item confrontare l'andamento dei nuovi positivi giornalieri tra Emilia Romagna e Lombardia
    \begin{enumerate}
        \item apertura dashboard;
        \item click su link alla sezione "Confronto tra regioni";
        \item selezione sulla mappa della regione "Emilia Romagna" e "Lombardia";
        \item lettura del grafico "Nuovi positivi";
    \end{enumerate}
    \item confrontare la media dei nuovi positivi giornalieri tra il mese di aprile e il mese di maggio
    \begin{enumerate}
        \item apertura dashboard;
        \item click su link alla sezione "Analisi di periodi";
        \item selezione del mese di Aprile nel primo widget calendario;
        \item selezione del mese di Maggio nel secondo widget calendario;
        \item lettura dei valori nel box "Nuovi positivi";
    \end{enumerate}
\end{enumerate}
\subsection{Scelta dei partecipanti}
\label{ss:scelta-partecipanti}
Per la scelta dei partecipanti, abbiamo cercato contatti di soggetti afferenti alla segmentazione dell'utenza definita nella sezione "Ricerca etnografica". Un criterio che abbiamo seguito è la diversificazione della redazione di riferimento (testata locale, nazionale, d'agenzia ecc.), al fine di avere risultati privi di bias.

\subsection{Esecuzione dei test}
\label{ss:esecuzione-test}
Il test è stato eseguito sui wireframe della riprogettazione della dashboard del DPC: trattasi di un mock-up che offre diverse possibilità di interazione all'utente, per permettergli di comprendere il funzionamento di ogni componente grafica.\\
L'approccio seguito è stato quello del "Thinking aloud" per cui abbiamo chiesto a ciascun giornalista di eseguire task pre-determinati sull'interfaccia e di riferirci, appunto a voce alta, le sue impressioni e intenzioni step by step.\\

-- INTERVISTA --

\subsection{Valutazione finale e report}
\label{ss:valutazione-finale-report}

-- DA SCRIVERE --